\chapter{Description des livrables}

Plusieurs livrables doivent être livrés aux clients tout au long des différentes étapes de l'étude préalable :

	\section{Initialisation et Organisation du projet}

		Deux livrables sont à rendre lors de cette étape afin de fixer le contexte et l'objet du projet et assurer des méthodes et des processus qualité robustes et ainsi assurer des livrables de qualité pour les prochaines étapes.

		\subsection{Dossier d'initialisation}

		Le dossier d'initialisation sert de point de départ pour le projet et doit contenir les éléments suivants :

		\begin{enumerate}
			\item Objet et contexte du projet
			\item Description des livrables attendus
			\item Modes opératoires et phasage
			\item Planning et définition des tâches
			\item Organsation de l'équipe
		\end{enumerate}

		\subsection{Plan d'Assurance Qualité (PAQ)}

		Ce document permet la mise en place d'une politique qualité pour le projet. Il doit contenir :

		\begin{enumerate}
			\item Spécification de la forme (plan, mise en forme) des documents
			\item Cycle de vie des documents
			\item Ressources et outils utilisés
			\item Politique qualité et méthodologie de validation interne et de recette
		\end{enumerate}


	\section{Expression des besoins}

		Cette étape permet de produire, en collaboration avec les clients, le dossier d'expression des besoins.

		\subsection{Dossier d'expression des besoins}

		 Ce dossier résult du receuil les besoins clients et l'étude de l'existant / benchmarking. Il contient :

		\begin{itemize}
		    \item une présentation du contexte du projet (approche métier),
		    \item les éventuelles orientations stratégiques de la MOA
		    \item une analyse de l’existant (dont le SI)
		    \item la cible fonctionnelle (modèle de référence des activités et processus de l’entreprise).
		    \item les écarts avec l’existant (les dysfonctionnements)
		    \item les attentes des partenaires
		    \item le benchmarking
		    \item les thèmes de progrès
		\end{itemize}

	\section{Expression des besoins}

		Cette étape permet de produire, en collaboration avec les clients, le dossier d'expression des besoins.

		Ce dossier résult du receuil les besoins clients et l'étude de l'existant / benchmarking. Il contient :
		\begin{itemize}
		    \item Présentation du contexte Métier du projet
		    \item Orientations stratégiques de la MOA
		    \item Analayse de l’existant
		    \item Cible fonctionnelle : activités et processus de l’entreprise
		    \item Écarts entre la cible et l’existant (manques, dysfonctionnements)
		    \item Attentes des partenaires
		    \item Benchmarking : étude des solutions utilisées dans 2 entreprises du secteur
		    \item Thèmes de progrès
		\end{itemize}


	\section{Élaboration des scénarios}

		Durant cette étape doivent être étudiés et envisagés deux scénarios de mise en œuvre : une solution spécifique et une solution ERP (standard).

		Ce document doit contenir (pour chaque scénario):
		\begin{itemize}
		    \item Nouvelle organisation
		    \item Architecture technique
		    \item Architecture applicative
		    \item Architecture logicielle
		\end{itemize}


	\section{Évaluation des scénarios}

		Les deux scénarios établis précédemment sont évalués lors de cette étape et leurs principaux avantages/inconvénients sont comparés afin de faire un choix.

		Un livrable devra reprendre cette étude en explicitant le choix effectué et en permettant au client de comprendre les raisons en quoi ce dernier répond à ses besoins.nts de choix, à savoir les points forts et les points faibles.


	\section{Restitution}

		Lors de la restitution, plusieurs livrables sont à prévoire: un dossier bilan et une présentation orale (type Powerpoint).

		\subsection{Présentation orale}

		Nous exposerons durant cette présentation les deux solutions étudiées (solution spécifique et solution ERP) et expliquerons notre choix, le but étant de convaincre le client du sérieux de notre étude préalable et de la qualité des livrables fournis.

		\subsection{Dossier bilan}

		Le dossier bilan est une synthèse de l'étude préalable et doit contenir:

		\begin{itemize}
		    \item Les changements qu'ont subi les livrables de la phase d'initialisation à la restitution
		    \item Synthèse des difficultés rencontrées.
		    \item Planning mis à jour
		    \item Justification des écarts entre le planning prévisionnel et le planning effectif.
		\end{itemize}



	\section{Documents de suivi}

		Un certain nombre de documents sont réalisés tout au long de la réalisation de l'étude préalable et sont utiles en interne pour superviser la réalisation des tâches et l'état de l'équipe.

		\begin{itemize}
		    \item Fiche de suivi individuel par séance
		    \item Fiche de suivi global par séance
		    \item Fiche de suivi d’avancement des livrables intermédiaires
		    \item Journal de réunion
		    \item Tableau de bord
		\end{itemize}