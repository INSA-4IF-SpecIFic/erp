% REMOVE THIS
\documentclass[a4paper,11pt]{report} 

\title{Dossier d'Initialisation}

\date\today

\begin{document}
\maketitle
\newpage

% END REMOVE THIS

\chapter*{M\'ethodes et phasage}
\addcontentsline{toc}{chapter}{M\'ethodes - Modes op\'eratoires - Phasage}
\chaptermark{M\'ethodes - Modes op\'eratoires - Phasage}

\subsection*{M\'ethodes utilis\'ees}

M\'ethodes...

\subsection*{Phases}

Nous diviserons ce projet en 4 grandes phases :


\begin{itemize}
 \item Organisation du projet
 \item Expression des besoins
 \item Construction des solutions
 \item \'Evaluation des sc\'enarii
\end{itemize}


Chacune de ces phases comporte des objectifs pr\'ecis.

\subsubsection*{Organisation du projet}

Afin d'effectuer ce projet dans les meilleures conditions, il est primordial d'adopter une organisation efficace.
Dans un premier temps, il est important de resituer l'\'etude dans son contexte. 
Ceci est n\'ecessaire afin de ne pas perdre de temps \`a \'elaborer des solutions qui s'avèreront \^etre hors du champs de l'\'etude demand\'ee.
Notre \'equipe prendra donc connaissance du projet finement et identifiera sa place dans les activit\'es de l'entreprise.

L'ensemble des livrables \`a fournir devra \^etre identifi\'e afin de d\'eterminer les diff\'erentes t\^aches \`a r\'epartir au sein de l'\'equipe.
Nous identifierons ensuite les contraintes et les risques li\'es \`a ce projet et \'etablirons des plans d'actions pour les g\'erer.

Afin de s'assurer de la qualit\'e des livrables, un plan d'assurance qualit\'e (PAQ) sera mis en place. Dans celui-ci figurera la gestion de la documentation du projet,
le workflow, les proc\'edures de validations internes et externes, ainsi que l'ensemble des outils que nous utiliserons.

\subsubsection*{Expression des besoins}

Cette phase d'expression des besoins consiste \`a faire tout d'abord une \'etude de l'existant.
Il s'agit d'\'etudier les moyens organisationnels et informatiques d\'ej\`a mis en place dans l'entreprise afin de dimensionner les changements \`a effectuer concernant les processus et les strat\'egies de l'entreprise.
Nous ciblerons notre étude aux processus de \textbf{gestion des contrats de maintenance}.

Après ceci vient une étape de benchmarking dans laquelle il faudra établir une comparaison entre l'entreprise et ses concurrents en termes \'economique et d'activités, de SI et de modèles standards de processus.
Pour cela il nous faudra choisir des indicateurs permettant l'évaluation des performances de l'entreprise. Les résultats du benchmarking donnerons des ``bonnes partiques'' à suivre pour booster les performances de l'entreprise.

A partir des résultats du benchmarking, des dysfonctionnements détectés et des  attentes des clients, nous construirons des modèles conceptuels du futur SI à mettre en place.

\subsubsection*{Construction des solutions}

Dans le cadre de ce projet, nous élaborerons deux solutions :

\begin{itemize}
 \item une solution \textit{spécifique} répondant aux attentes organisationnelles et informatiques de la gestion des contrats de maintenance. Cette solution visera à répondre le plus précisement possible aux besoins.
 \item une solution \textit{standard} identifiant les éléments de réferentiel standard pour la gestion des contrats de maintenance. Elle sera basée sur une solution ERP standard.
\end{itemize}

\subsubsection*{\'Evaluation des sc\'enarii}

A la suite de ces phases, nous soutiendrons ce projet devant un jury.

\subsection*{R\'esultats des phases}

Chaque phase engendre des livrables.

% REMOVE THIS
\end{document}
% END REMOVE THIS
