\chapter*{M\'ethodes et phasage}
\addcontentsline{toc}{chapter}{M\'ethodes - Modes op\'eratoires - Phasage}
\chaptermark{M\'ethodes - Modes op\'eratoires - Phasage}

\subsection*{M\'ethodes utilis\'ees}
\addcontentsline{toc}{subsection}{M\'ethodes utilis\'ees}


M\'ethodes...

\subsection*{Phases}
\addcontentsline{toc}{subsection}{Phases}

Nous diviserons ce projet en 4 grandes phases :


\begin{itemize}
 \item Organisation du projet
 \item Expression des besoins
 \item Construction des solutions
 \item \'Evaluation des sc\'enarii
\end{itemize}


\subsubsection*{Organisation du projet}

Afin d'effectuer ce projet dans les meilleures conditions, il est primordial d'adopter une organisation efficace.
Dans un premier temps, il est important de situer correctement l'\'etude dans son contexte. 
Ceci est n\'ecessaire afin de ne pas perdre de temps \`a \'elaborer des solutions qui s'av\`ereront \^etre hors du champs de l'\'etude demand\'ee.
Notre \'equipe prendra donc connaissance du projet en détails et identifiera sa place dans les activit\'es de l'entreprise.

L'ensemble des livrables \`a fournir devra \^etre identifi\'e afin de d\'eterminer les diff\'erentes t\^aches \`a r\'epartir au sein de l'\'equipe.
Nous identifierons ensuite les contraintes et les risques li\'es \`a ce projet et \'etablirons des plans d'actions pour les g\'erer.

Afin de s'assurer de la qualit\'e des livrables, un plan d'assurance qualit\'e (PAQ) sera mis en place. Dans celui-ci figurera la gestion de la documentation du projet,
le workflow, les proc\'edures de validations internes et externes, ainsi que l'ensemble des outils que nous utiliserons.

\subsubsection*{Expression des besoins}

Cette phase d'expression des besoins consiste \`a faire tout d'abord une \'etude de l'existant.
Il s'agit d'\'etudier les moyens organisationnels et informatiques d\'ej\`a mis en place dans l'entreprise afin de détecter les dysfonctionnements et de dimensionner les changements \`a effectuer concernant les processus et les strat\'egies de l'entreprise.

Nous ciblerons notre \'etude aux processus de \textbf{gestion des contrats de maintenance}.

Apr\`es ceci vient une \'etape de benchmarking dans laquelle il nous faudra \'etablir une comparaison entre l'entreprise et ses concurrents en termes \'economiques, d'activit\'es, de SI et de mod\`eles standards de processus.
Pour cela il nous devrons choisir des indicateurs permettant l'\'evaluation des performances de l'entreprise. Les r\'esultats du benchmarking donnerons des bonnes partiques \`a suivre pour booster les performances de l'entreprise.

A partir des r\'esultats du benchmarking, des dysfonctionnements d\'etect\'es et des  attentes des clients, nous construirons des mod\`eles conceptuels du futur SI \`a mettre en place.

\subsubsection*{Construction des solutions}

Dans le cadre de ce projet, nous \'elaborerons deux solutions :

\begin{itemize}
 \item une solution \textit{sp\'ecifique} r\'epondant aux attentes organisationnelles et informatiques de la gestion des contrats de maintenance. Cette solution visera \`a r\'epondre aux besoins le plus pr\'ecisement possible.
 \item une solution \textit{standard} identifiant les \'el\'ements de r\'eferentiel standard pour la gestion des contrats de maintenance. Elle sera bas\'ee sur une solution ERP standard.
\end{itemize}

\subsubsection*{\'Evaluation des sc\'enarii}

Nous effectuerons une \'etude comparative des diff\'erents sc\'enarii propos\'es afin d'en faire ressortir les avantages et les d\'efauts de chacun.
Nous chifferons les diff\'erentes solutions et d\'eterminerons les gains et les ROI.
Nous aiderons ensuite le client \`a choisir la solution la plus adapt\'ee \`a ses besoins.


\subsection*{R\'esultats des phases}
\addcontentsline{toc}{subsection}{R\'esultats des phases}

Chacune de ces phases engendrera des livrables.

\begin{itemize}
 \item Organisation du projet : un dossier d'initialisation et un PAQ
 \item Expression des besoins : un dossier d'expression des besoins et un rapport de mod\'elisation
 \item Construction des solutions : un dossier de description des sc\'enarii
 \item \'Evaluation des sc\'enarii : un dossier de choix
\end{itemize}

A ces livrables s'ajouteront des livrables interm\'ediaires permettant un suivi plus fin du d\'eroulement du projet.
