\chapter{Cibles de fonctionnement}


\section{Axes d'amélioration}

    Les chapitres précédents --- analyse de l'existant et benchmarking -- permettent d'identifier quelques axes sur lesquels le système d'information de SPIE péche contre les bonnes pratiques et où des améliorations sont à prévoir.

    Nous allons donc citer dans ce document certains thèmes d'amélioration possibles à travers les axes suivant:

    \begin{enumerate}
        \item Identification de nouvelles technologies à forte valeur ajoutée~;
        \item Réorganisation des acteurs des processus métiers de SPIE pour suivre les ``best-practices'' dégagées~;
        \item Réorganisation de la logique des processus existants~;
    \end{enumerate}


    \subsection{Nouvelles technologies}

    L'intégration de nouvelles technologies peut permettre à SPIE de se démarquer de la concurrence en changeant de manière radicale la gestion de certaines opérations.

    Nous allons nous intéresser à certaines technologies que SPIE pourrait intégrer~:

        \subsubsection{Gestion de la Relation Client avec un CRM}

        Quand on s'intéresse à la cartographie générale du SI de SPIE, la première chose qu'on remarque est l'absence d'une gestion de la Relation Client en utilisant un outil dédié, de type CRM.

        Les outils CRM (\textit{Customer Relationship Management}, ou Gestion de la Relation Client) permettent une gestion optimale la Relation Client avec des outils de modélisation, de reporting et de prédiction.

        L'utilisation d'outils dédiés permet autant d'améliorer les processus de Relation Client que de prédire les attentes de clients et ainsi prévenir les échecs et mieux s'y préparer.

        Les outils CRM permettent en outre de mener des campagnes de prospection, de suivre les opportunités commerciales et de fidéliser les clients.

        On peut citer à titre d'exemple \textit{Salesforce}, l'un des leaders du marché qui propose un CRM en mode \textit{SaaS}.




    Nous allons maintenant nous intéresser à la mise en place d'outils récents à forte valeur ajoutée qui ne sont pas nécessairement déployés chez les autres acteurs du secteur et peuvent apporter à GSTP un avantage concurrentiel.



* Donner aux techniciens un appareil qui permet de reporter les applications faites sur place
* Affecter un (ou des) techniciens à chaque client. Le technicien connaîtra déjà l'infrastructure du client à chaque intervention ce qui devrait accélérer les choses et donner un meilleur processus
