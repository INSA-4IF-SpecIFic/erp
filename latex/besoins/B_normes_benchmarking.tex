\chapter{Normes et benchmarking}
\section{Benchmarking}
\subsection{Introduction}
	Le Benchmarking a pour but d'optimiser une ou plusieurs fonctions de l'entreprise. Pour cela nous allons analyser les contextes métier et technique de ses principaux concurrents, réputés comme étant les meilleurs, ou leader dans un domaine commun à SPIE. Il faut se rapprocher de ce qui semble le plus parfait à l'heure actuelle.
	Les buts de cette étude sont~:
\begin{itemize}
	\item la compréhension de leurs méthodes~;
	\item la mise en place d'indicateurs (quantité, délais, coûts)~;
	\item des chiffres de références pour se situer par rapport à leurs performances~;
	\item une inspiration pour créer de meilleures pratiques à partir des leurs.
\end{itemize}
\bigbreak
	Grâce à toutes ces connaissances, le benchmarking va nous permettre de~:
\begin{itemize}
	\item de mettre en place de meilleures pratiques (Best-practices)~;
	\item de créer des activités et processus types~;
	\item de pouvoir étudier l'évolution de l'entreprise grâce aux chiffres des performances des concurrents.
\end{itemize}
\bigbreak
	Dans un premier temps, nous allons donc détailler quelques entreprises leader dans les domaines communs à SPIE. Puis nous nous intéresserons aux systèmes informatiques disponibles sur le marché, ainsi que ceux utilisés par les concurrents.

    \section{Benchmarking des ERP existants}
		Il existe un nombre considérables d'ERPs, autant commerciaux que libres, existent dans le marché. Les coûts, le processus d'intégration et les méthodes utilisées pour modéliser les process métiers variant d'un ERP à l'autre, un benchmarking critique s'imposait afin d'extraire les solutions les plus adaptées aux besoins de SPIE.

    	Certains ERPs se prévalent d'une certaine généricité: adaptés aux petites comme aux grandes entreprises et à tous les secteurs et métiers. On peut citer la famille d'ERPs SAGE par exemple, qui propose des solutions généralistes. D'autres ERP comme SAP ByDesign restent adaptés à une grande variété de secteurs mais plutôt adaptés à des entreprises d'une certaine taille (PME dans le cas de ByDesign).

    	Dans la suite, nous allons voir plusieurs solutions ERP et analyser leurs fonctionnalités / modèles métiers.

	    \subsection{SAP}

	    SAP est une multinationale, dont le siège est à Walldorf en Allemagne, qui conçoit et commercialise des logiciels de gestion et de maintenance, principalement à destination des entreprises et institutions.

	    SAP a un chiffre d'affaire de 16,22 milliards € et est le plus gros concepteur de logiciels en Europe, et le 4ème à l'échelle mondiale. Tous les indicateurs montrent que l'entreprise est en bonne santé et ne risque pas de couper le support de ces logiciels.

	    Le produit phare de SAP est \textbf{SAP ERP}, un ERP hautement modulaire contenant un très grand nombre de modules.

	    Les modules ERP rentrent dans trois catégories principales: Logistique, Gestion comptable et Ressources humaines.


        %\ !!! Tout ce qu'il y a en dessous est tiré d'un autre projet ERP !!!

\subsection{Les concurrents, leaders du marché}
\subsubsection{Présentation rapide de SPIE}
	Leader européen dans les domaines de l'énergie et des communications, SPIE accompagne ses clients privés et publics dans la conception, la réalisation, l'exploitation et la maintenance d'installations plus économes en énergie et plus respectueuses de l'environnement.
	Sa mission : Apporter du changement, un progrès durable et une amélioration de la qualité du cadre de vie.
\newpage
	Pour cela, le Groupe accompagne les collectivités et les entreprises dans la conception, la réalisation, l'exploitation et la maintenance de leurs installations, grâce à son expertise dans les domaines tels que :
\begin{itemize}
	\item Génie électrique
	\item Génie mécanique
	\item Génie climatique
	\item Pétrole et gaz
	\item Nucléaire
	\item Energies renouvelables
	\item Systèmes de communication et infogérance
	\item Réseaux extérieurs et éclairage public
\end{itemize}
\bigbreak
Chiffres clés 2012 :
\begin{itemize}
	\item 30 200 collaborateurs
	\item 31 pays
	\item 500 implantations
	\item 4 217 millions d'euros de chiffres d'affaires (+ 4,3\%), répartis entre ses 4 segments stratégiques ("Energies" (26\%), "E-fficient building" (32\%), "Smart city" (25\%) et "Industry services" (17\%)).
\end{itemize}

\subsubsection{Les concurrents en général}
Vinci Energies (Siège social en France ; Chiffre d'affaire de 9 milliards d’euros en 2012)
Eiffage (Siège en social en France ; Chiffre d'affaire de 14 milliards d’euros en 2012)
	Nous allons étudier ici les best practices de deux de ces groupes : Vinci Energies et Eiffage.

\subsubsection{Vinci Energies}
	Le pôle Energies de VINCI, issu du rapprochement de VINCI Energies et de Cegelec en 2010 est leader en France et acteur majeur en Europe.
	Expert dans chacun de ses domaines technologiques de prédilection et expert du métier de ses clients, VINCI Energies bâtit à partir de leurs besoins des offres qui répondent à leur enjeux de performance, de fiabilité et de sécurité en :
\begin{itemize}
	\item Energie électrique
	\item Génie climatique et thermique
	\item Mécanique
	\item Technologies de l'information et de la communication
\end{itemize}
\bigbreak
	Pour cela il intervient depuis l'ingénierie et la réalisation, jusqu'à la maintenance, l'exploitation et le facility management. Il intervient au service des collectivités publiques et des entreprises pour déployer, équiper, faire fonctionner et optimiser leurs infrastructures d'énergie, de transport et de communication, leurs sites industriels et leurs bâtiments.
\newpage
Chiffres clés 2012 :
\begin{itemize}
	\item 64 00 collaborateurs
	\item 45 pays
	\item 9 milliards d'euros de chiffres d'affaires
\end{itemize}

\subsubsection{Eiffage}
	Leader européen des concessions et du BTP, Eiffage exerce ses activités à travers cinq métiers :
\begin{itemize}
	\item Concessions et partenariats public-privé (énergie, réseaux, grands ouvrages d'infrastructures autoroutières et ferroviaires...)
	\item Construction (facily management, bâtiment, immobilier...)
	\item Travaux publics (génie civil, route...)
	\item Energie (génie électrique, génie climatique...)
	\item Métal (maintenance industrielle...)
\end{itemize}
\bigbreak
	Si la notoriété du groupe provient de ses réalisations de référence, son engagement pour l'environnement et la sauvegarde de la biodiversité est mis en oeuvre sur l'ensemble de ses chantiers d'infrastructures. Aménager et construire tout en valorisant la qualité de vie de chacun est bien la marque d'Eiffage.
\bigbreak
Chiffres clés 2012 :
\begin{itemize}
	\item 70 00 collaborateurs
	\item 13 pays
	\item 14 milliards d'euros de chiffres d'affaires
\end{itemize}

\subsubsection{Les Best-practices relevées}
\begin{itemize}
	\item Définir le périmètre des interventions (choisir les prestations à mettre dans le contrat, définir la gestion des délais et dates, et choisir les modalités de paiement etc).
	\item Mesures des performances des prestations et mise en place de pénalité.
	\item Délimitation des garanties du contrat de maintenance.
	\item Assurer la traçabilité de l'exécution du contrat par la constitution d'une base de données de preuves.
	\item Possibilité d'utiliser un logiciel de contrat de maintenance tel que G-CONTRATS, ou le GMAO (Gestion de Maintenance Assistée par Ordinateur) de Technic-Soft. Ils permettent des renouvellements automatiques de contrats et de facturations, de planifier les visites, de surveiller les échéances etc.
	\item Possibilité de déléguer certaines activités de maintenance à des prestataires.
	\item Définir le paiement de ces prestataires.
\end{itemize}
