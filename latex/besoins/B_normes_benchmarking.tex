\chapter{Normes et benchmarking}

    \section{Benchmarking des ERP existants}
		Il existe un nombre considérables d'ERPs, autant commerciaux que libres, existent dans le marché. Les coûts, le processus d'intégration et les méthodes utilisées pour modéliser les process métiers variant d'un ERP à l'autre, un benchmarking critique s'imposait afin d'extraire les solutions les plus adaptées aux besoins de SPIE.

    	Certains ERPs se prévalent d'une certaine généricité: adaptés aux petites comme aux grandes entreprises et à tous les secteurs et métiers. On peut citer la famille d'ERPs SAGE par exemple, qui propose des solutions généralistes. D'autres ERP comme SAP ByDesign restent adaptés à une grande variété de secteurs mais plutôt adaptés à des entreprises d'une certaine taille (PME dans le cas de ByDesign).

    	Dans la suite, nous allons voir plusieurs solutions ERP et analyser leurs fonctionnalités / modèles métiers.

	    \subsection{SAP}

	    SAP est une multinationale, dont le siège est à Walldorf en Allemagne, qui conçoit et commercialise des logiciels de gestion et de maintenance, principalement à destination des entreprises et institutions.

	    SAP a un chiffre d'affaire de 16,22 milliards € et est le plus gros concepteur de logiciels en Europe, et le 4ème à l'échelle mondiale. Tous les indicateurs montrent que l'entreprise est en bonne santé et ne risque pas de couper le support de ces logiciels.

	    Le produit phare de SAP est \textbf{SAP ERP}, un ERP hautement modulaire contenant un très grand nombre de modules.

	    Les modules ERP rentrent dans trois catégories principales: Logistique, Gestion comptable et Ressources humaines.


        %\ !!! Tout ce qu'il y a en dessous est tiré d'un autre projet ERP !!!
		Les modules sont les composants fonctionnels du système SAP ERP. On peut distinguer trois familles de modules fonctionnels : logistique, gestion comptable et ressources humaines.

	    \subsection{SAGE X3 Edition : un ERP générique visant un large public}

		    %\subsubsection{Offre Moyennes et grandes entreprises}
		        %\url{http://www.sage.fr/mge/logiciels-erp}
		        %\subsubsection{Offre PME}
		        %\url{http://www.sage.fr/pme/logiciels-de-gestion/erp}

		Entreprise de plus de 20 ans, Sage a tissé des liens pérennes avec les acteurs stratégiques du bâtiment, les organisations professionnelles, les industriels, les négociants en matériaux, les centres de formation, les experts-comptables, etc.
		Cet ERP propose 2 offres pour le secteur du BTP, l'une s'adressant au PME, l'autre au plus grande entreprise.

		L'offre SAGE 100 Multi Devis propose de répondre aux besoins suivants :
		\begin{itemize}
		    \item Saisir simplement les devis
		    \item Réaliser des études de prix complexe
		    \item Gérer le déboursé, le prix de revient et maîtriser les marges
		    \item Préparer automatiquement les factures et les situations intermédiaires
		    \item Gérer les cycles d’achats et la relation avec les fournisseurs
		    \item Contrôler la gestion du temps par salariés et par chantier
		    \item Assurer la gestion comptable et financière de votre entreprise
		    \item Gérer votre personnel en conformité avec les obligations légales\\
		\end{itemize}

		La section gestion du parc matériel semble peu présente, voire inexistante.

		\subsubsection{SAP}

		SAP ERP est composé d'une centaine de modules fonctionnels bien précis (Material Management, Sales and Distribution,... ).
		Le principal intérêt de SAP ERP est qu'il est totalement flexible. On peut installer tous les modules fonctionnels, ou seulement quelques-uns.
		Aucun superflux. SAP ERP est entièrement paramétrable et s'adapte ainsi aux besoins et à la structure de l'entreprise.
		Grâce à ses fonctionnalités, ce progiciel s'adapte parfaitement au secteur du BTP.
		Enfin, grâce à son environnement de développement, SAP ERP peut être adapté à des besoins spécifiques.\\
		\paragraph{Description du module le plus intéressant : Material Management}
		Le module MM (Material Management) concerne la gestion des articles d'un point de vue achats et gestion des stocks.
		Y sont intégrées des notions telles que :
		\begin{itemize}
		    \item Le calcul des besoins, des réapprovisionnements (MRP - Material Requirements planning)
		    \item La gestion des achats
		    \item contrats, demandes d'achats, etc.
		    \item commandes de biens, de services
		    \item Mouvements de stocks
		    \item réceptions de marchandises
		    \item Valorisation des stocks en intégration avec FI
		    \item Contrôle des factures
		    \item Gestion des stocks
		    \item entrées, sorties, transferts de stocks
		    \item Gestion des emplacements magasin (WM Warehouse Management)
		    \item Inventaire
		\end{itemize}

		D'autres modules tel que le module SD (Sales and Distribution) pourront nous intéresser.
		SAP ERP peut nous permettre de répondre exactement aux exigences du client. Mais est-il pertinent d'utiliser une telle usine à gaz pour une PME?

		\subsubsection{PROGIB}
		%\url{http://www.quelsoft.com/fiche/progib-m46-43-148.html}

		Progiciel intégré pour les entreprises du bâtiment, des travaux publics et d'espaces verts.\\

		Le noyau central est l'analyse en temps réel de la rentabilité et de l'avancement des chantiers et des affaires. A celà s'ajoute des modules complémentaires tels que
		: gestion de stocks avec codes à barres, comptoir de vente, parcs matériels, suivi des contrats de maintenance et petits dépannages, CRM, module décisionnel.
		Un des atouts de PROGIB est sa toute dernière nouveauté : le suivi des interventions sur des Pockets-PC et la synchronisation par GPRS.

                \subsubsection{ONAYA}
                ONAYA est un outil de gestion issu de 20 annees de travail du groupe Aquitaine Informatique, en collaboration avec les entreprises de travaux publics. Ce progiciel de gestion integre (ERP/PGI) est lui aussi une application modulaire permettant de gerer et piloter une entreprise de travaux publics. Cette solution gere : les etudes de prix - devis, la facturation, le suivi de chantiers, la
logistique, la saisie nomade, le planning, la comptabilite et la paye. Cette solution couvre quasiment l'ensemble des besoins de gestion et de suivi des chantiers de GSTP (de la DM).

                                \paragraph{Gestion des chantiers}
                                \begin{itemize}
                                    \item Gestion de la nomenclature, production.
                                    \item Gestion des achats.
                                    \item Gestion des matériels.
                                    \item Gestion du personnel.
                                \end{itemize}

                                \paragraph{Gestion des achats}
                                \begin{itemize}
                                    \item Prise en compte des demandes d'approvisionnement pour un ou plusieurs chantiers.
                                    \item Consultation des fournisseurs.
                                    \item Etablissement des commandes fournisseurs ou réservation sur stock.
                                \end{itemize}

                                \paragraph{Gestion des stocks}
                                \begin{itemize}
                                    \item Approvisionnement du stock (commandes et réception des matériaux).
                                    \item Approvisionnement des chantiers.
                                    \item Statistiques de consommation.
                                \end{itemize}

                                \paragraph{Gestion des matériels}
                                \begin{itemize}
                                    \item Gestion du parc matériel.
                                    \item Gestion de l'atelier.
                                    \item Gestion des pièces de rechange.
                                \end{itemize}

                                \paragraph{Gestion du planning}
                                \begin{itemize}
                                    \item Plan de charges.
                                    \item Planning financier.
                                    \item Découpage et planification des ressources.
                                \end{itemize}

                Cette solution est très complète. Certaines parties sont trop détaillées, offrent beaucoup plus de fonctionnalités que ce dont nous avons besoin pour couvrir toute l'activité de GSTP.

                \subsubsection{BRZ 7 : ancien Kyetos2}

                BRZ 7 est un progiciel intégrée qui permet aux PME BTP d'unifier l'ensemble de leur processus métiers (Etude de prix, Gestion de chantier, gestion financière) autour d'une base de données unique.
    L'absence de ressaisie apporte des gains de productivité important.
    Cette solution améliore la performance des utilisateurs en leur offrant des outils d'aide à la gestion: Reporting analytique,
    planification et logistique chantier, gestion documentaire...\\
                BRZ 7 est composé de :

     \begin{itemize}
                  \item \textbf{Pointage Smartphone}
                  \item Gestion commerciale
                  \item Etude de prix et Risk management
                  \item Logistique et planification
                  \item Suivi de chantier
                  \item Gestion des achats et stocks
                  \item Gestion des sous-traitants
                  \item Comptabilité Générale et Analytique
                  \item Gestion des immobilisations
                  \item Contrôle de gestion, états financiers et fiscaux
                  \item Paie et Gestion et ressources humaines\\
                \end{itemize}

                BRZ 7 propose aussi plusieurs progiciels séparés. L'un d'entre eux peut nous intéresser. Il s'agit de Phenos.
    Thenos est un outils entièrement personnalisable pour une gestion adaptée à vos besoins :
    gestion technique, approvisionnement, stock, interventions, planification/logistique, suivi, calcul de rentabilité et intégration comptable.


\subsection{Comparatif des principaux ERP du batiments}

Nous avons ci-dessous représenter les avantages et les inconvénients des solutions ci-dessus.


\comparatif{SAP ERP}
{
    SAP ERP est composé d'une centaine de modules fonctionnels bien précis (Material Management, Sales and Distribution,... ).
    Le principal intérêt de SAP ERP est qu'il est totalement flexible. On peut installer tous les modules fonctionnels, ou seulement quelques-uns.
    Aucun superflux. SAP ERP est entièrement paramétrable et s'adapte ainsi aux besoins et à la structure de l'entreprise.
    Grâce à ses fonctionnalités, ce progiciel s'adapte parfaitement au secteur du BTP.
    Enfin, grâce à son environnement de développement, SAP ERP peut être adapté à des besoins spécifiques.
}
{
    \begin{itemize}
        \item Modulable
        \item Evolutif
        \item Adaptable au BTP
        \item Une forte expérience
    \end{itemize}
}
{
    \begin{itemize}
        \item Une usine à gaz
        \item Application lourde
    \end{itemize}
}



\comparatif{Onaya}
{
    Onaya est un ERP spécialement conçu pour les entreprises du bâtiment et des travaux publics (BTP).
    Onaya est aussi modulaire mais il ne propose que peu de modules (8).
    L'implantation de cet ERP demandera un changement sur l'ensemble de l'entreprise et pas seulement sur la DM.
}
{
    \begin{itemize}
        \item Modulable
        \item Propre au BTP
    \end{itemize}
}
{
    \begin{itemize}
        \item Peu de modules
        \item Remaniement de l'entreprise
    \end{itemize}
}

\comparatif{BRZ 7}
{
    Logiciel pour la gestion globale des entreprises de BTP. Constitué de neuf modules : étude de prix, métré, planning, facturation, suivi de chantier, gestion des achats, comptabilité générale et analytique et pointage main d'\oe uvre.
    Configuration mono ou multiposte. Possibilité d'installation en réseau (internet ou intranet) pour applications multisites.
}
{
    \begin{itemize}
      \item Pointage Smartphone
      \item Propre au BTP
      \item Multisites
    \end{itemize}
}
{
    \begin{itemize}
        \item Peu connu
        \item Maintenance%?
    \end{itemize}
}

\comparatif{SAGE}
{
        Troisième éditeur mondial de logiciels de gestion, Sage simplifie et automatise la gestion et les processus métier de 6,1 millions d'entreprises dans 70 pays à travers le monde.
        Sage propose une offre complète couvrant les besoins de toutes les entreprises.
        Sage a fait le choix d'une approche décentralisée : chacune de ses 26 filiales dispose d'une autonomie
        de décision et développe localement ses produits afin de répondre avec réactivité aux besoins spécifiques de chaque pays.
        Le développement en France des solutions destinées aux entreprises françaises permet  à  Sage de mettre rapidement à la disposition de ses clients des logiciels conformes à la réglementation locale.
        Cette stratégie permet à Sage de répondre dans des délais très courts  à  l'évolution des besoins des clients : les innovations sont le fruit de l'observation de ces besoins et de l'évolution de la réglementation.
        Sage met rapidement à la disposition des entreprises françaises les outils adaptés à leur croissance.
        Pour s'assurer de l'adéquation de son offre, Sage a mis en place une série d'indicateurs qui sont autant de critères de performance pour ses collaborateurs.
}
{
    \begin{itemize}
        \item International
        \item Développement local
        \item Démarche Qualité
        \item Maintenance
    \end{itemize}
}
{
    \begin{itemize}
        \item Une usine à gaz%?
        \item Remaniement Complet
    \end{itemize}
}

\comparatif{Progib}
{
    PROGIB s'appuie sur une équipe et une expérience de 25 ans dans l'informatique de gestion et s'est très fortement spécialisée dans les dernières années.
    Aujourd'hui la cible unique de l'entreprise est l'entreprise de bâtiment, travaux publics et espaces verts.
    PROGIB traite sa clientèle en direct mais propose également des services de proximité sur l'ensemble du territoire
    français par l'intermédiaire de son réseau de distributeurs agréés.
}
{
    \begin{itemize}
        \item Modulable
        \item Propre au BTP
        \item Multisites
        \item Pocket-PC
    \end{itemize}
}
{
    \begin{itemize}
        \item Noyau principal financier
        \item Maintenance%?
    \end{itemize}
}
