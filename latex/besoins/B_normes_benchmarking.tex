\chapter{Normes et benchmarking}
\section{Benchmarking}
\subsection{Introduction}
	Le Benchmarking a pour but d'optimiser une ou plusieurs fonctions de l'entreprise. Pour cela nous allons analyser les contextes métier et technique de ses principaux concurrents, réputés comme étant les meilleurs, ou leader dans un domaine commun à SPIE. Il faut se rapprocher de ce qui semble le plus parfait à l'heure actuelle.
	Les buts de cette étude sont~:
\begin{itemize}
	\item la compréhension de leurs méthodes~;
	\item la mise en place d'indicateurs (quantité, délais, coûts)~;
	\item des chiffres de références pour se situer par rapport à leurs performances~;
	\item une inspiration pour créer de meilleures pratiques à partir des leurs.
\end{itemize}
	Grâce à toutes ces connaissances, le benchmarking va nous permettre de~:
\begin{itemize}
	\item de mettre en place de meilleures pratiques (Best-practices)~;
	\item de créer des activités et processus types~;
	\item de pouvoir étudier l'évolution de l'entreprise grâce aux chiffres des performances des concurrents.
\end{itemize}

	Dans un premier temps, nous allons donc détailler quelques entreprises leader dans les domaines communs à SPIE. Puis nous nous intéresserons aux systèmes informatiques disponibles sur le marché, ainsi que ceux utilisés par les concurrents.
