\chapter{Analyse de l’\'existant}

\section{Existant Organisationnel}

\section{Processus strat\'egiques}

Le processus de gestion de contrats de maintenance et services est divis\'e en sept sous-processus.

\begin{itemize}
    \item Offre et revue d'offre
    \item Commande et revue de commande
    \item Lancements des prestation de service et travaux
    \item R\'ealisation de prestations de maintenances
    \item R\'ealisation travaux induits
    \item Evolution du contrat
    \item Solde de l'affaire et du contrat
\end{itemize}

\subsection{Offre et revue d'offre}

Le sous-processus Offre et revue d'offre se divise en etapes majeurs~:

\begin{itemize}
    \item On commance tout d'abort l'etudie des contrat potentiel a partir de processus commercial,
    processus tarvaux ou appel d'offre. Avant la prise de decision quand a la poursuite ou non,
    il est necessaire de recuperer l'ensemble des informations. Le processus peut s'arreter ici
    en cas d'abandon.
    \item Une reponse (d'une appel d'offre par exemple) est fournie au client potentiel simplement
    pour bute que SPIE entre officielement dans la course \'a une proposition d'une solution. Alors
    commence l'etude du chiffrage, choix et validation de la solution qui sera propos\'e au client.
    \item Ensuite, l'offre initiale est r\'ealiser en interne a partir de la solution choisie, puis
    valid\'e sous la responsabilit\'e du Pilote de l'Offre.
    \item Enfin cette offre initiale est envoy\'ee au client avec courrier d'accompagnement dans les
    Delais impartie.
\end{itemize}

\subsection{Commande et revue de commande}

Le sous-processus Commande et revue de commande est l'etape comporte l'etape importante de negociation
avec le client et se divise en etapes majeurs~:

\begin{itemize}
    \item L'offre est donc envoy\'ee au client, celle ci est alors enregistr\'e, diffus\'e puis le
    porteur attritr\'e est design\'e par le Responsable Activit\'e.
    \item Ensuite, pour pr\'epar\'e la negociation avec le client, arrive la commission de revue de
    commande puis validation des ecarts de negociation, plan d'action de validation.
    \item N\'egociation avec le client.
    \item Acceptation ou refus de la commande d\'efinitive tout juste n\'egoci\'ee avec le client.
    En cas de refus, le processus est alors abandonn\'e.
    \item Redaction et signature du contrat entre SPIE et le client.
\end{itemize}

\subsection{Lancements des prestation de service et travaux}

Le sous-processes Lancements des prestation de service et travaux se divise en etapes majeurs~:

\begin{itemize}
    \item Prise en compte du dossier contractuel et du dossier d'etude par le Responsable d'Affaire.
    \item Realisation du dossier complet et analyse r\'ealis\'ee.
    \item Hello
\end{itemize}

\subsection{R\'ealisation de prestations de maintenances}

Le sous-processes Lancements des prestation de service et travaux se divise en etapes majeurs~:

\begin{itemize}
    \item Hello
\end{itemize}

\subsection{R\'ealisation travaux induits}

\begin{itemize}
    \item Hello
\end{itemize}

\subsection{Evolution du contrat}

\begin{itemize}
    \item Hello
\end{itemize}

\subsection{Solde de l'affaire et du contrat}

\begin{itemize}
    \item Hello
\end{itemize}

\section{Existant informatique}

\section{Dysfonctionnements}

