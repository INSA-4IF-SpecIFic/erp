\chapter{Attentes client}

	Nous avons été contactés par la société SPIE Sud-Est afin d'améliorer certains points dans le processus de gestion des contrats de maintenance. Il s'agit de proposer des solutions pour :

    \begin{itemize}
        \item développement de procédures métier et de supports d'exploitation par les entités de maintenance et service
        \item standardisation des procédures métier et de supports d'exploitation pour les entités exerçant le même métier sur le même secteur d'activité client
        \item analyse des risques propres à chaque métier sur le même secteur d'activité client
        \item amélioration de la définition des limites des interfaces avec les autres processus
        \item mise en place d'un Info centre dur l'intranet
    \end{itemize}

    Pour cela, nous proposons des interventions sur différents niveaux :


\section{Les procédures}

Après analyse des processus métier fournis par SPIE, on remarque que la Direction Générale (DG) n'intervient
pas dans le processus gestion de contrat et de maintenance et services.
C'est là alors un problème qui devrait être rectifié à l'avenir. En effet, il est vital que les acteurs intervenants dans ces processus, encadrant ou non, soient eux aussi encadrés.
Par ailleurs, cela permettrait de solidariser l'ensemble des contrats de SPIE, permettant de repartir les ressources physique et humaine de manières plus égales à travers ces derniers.
Cela impliquerait alors la création d'un processus de revues de l'ensemble des autres processus en cours d’exécution, ou terminés.

Un dernier problème récurent dans les processus actuels est qu'il n'y aucune étape dédiée au retour d'expérience, que ce soit du client, ou en interne. Pour connaître les cause et conséquences de chaque faiblesses étant survenues et les rectifier, nous rajouterions alors deux étape dans le sous-processus Solde de l'Affaire et du Contrat~:

\begin{itemize}
    \item Un pour les retours client afin de pouvoir \textit{brainstormer} ensuite sur des solutions aux problèmes ayant
    pu survenir~;
    \item L'autre pour les internes des différents acteurs. La Direction Générale aurait ici sa place en cas
    de décision de force majeur (comme par exemple le renvoi d'un employer aillant fait une faute professionnelle).
\end{itemize}


\section{Des nouvelles technologies}

L'intégration de nouvelles technologies peut permettre à SPIE de se démarquer de la concurrence en changeant de manière radicale la gestion de certaines opérations.

Nous allons nous intéresser à certaines technologies que SPIE pourrait intégrer~:


    \subsubsection{Gestion de la Relation Client avec un CRM}

        Les outils CRM (\textit{Customer Relationship Management}, ou Gestion de la Relation Client) permettent une gestion optimale la Relation Client avec des outils de modélisation, de \textit{reporting} et de prédiction.

        L'utilisation d'outils dédiés permet autant d'améliorer les processus de Relation Client que de prédire les attentes de clients et ainsi prévenir les échecs et mieux s'y préparer. Là où SPIE utilise actuellement plusieurs outils (\textit{Clarify}, \textit{ADV}, \textit{SUPRA}) aux infrastructures parfois séparées, l'utilisation d'un seul CRM simplifierai les processus de facturation, de suivi contrats et autres processus clients, et permettrait une meilleure intégration entre ces derniers.

    \subsubsection{SI utilisable pendant les interventions (application mobile)}

        Les techniciens ne peuvent actuellement pas utiliser les applications SPIE pendant une intervention: ils ne disposent pas de tablettes fournies à ce but et les applications SPIE ne sont pas optimisés (ou même parfois utilisables) sur des terminaux mobiles.

    \subsubsection{Moteur de recommandation et de préventions des risques}

        SPIE peut valoriser ses années d'expérience dans le domaine de la maintenance en créant un moteur de recommandation exploitant toutes les données récoltées lors d'interventions passées afin de proposer aux techniciens des recommandations similaires à celle qu'ils souhaitent effectuer.

\section{L'organisation}

\section{Les indicateurs}

    Nous avons dégagés des indicateurs permettant de suivre de manière quantitative la qualité et l'efficacité des processus de SPIE Sud-Est.

    \begin{description}
        \item[Gestion des comptes~:] ~ \\
            \begin{itemize}
                \item Détails du compte~: Synthèse des données principales du compte (id, adresse, données bancaires, ...)~;
                \item Historique d'activité du compte~: Nombre de rendez-vous, appels téléphoniques, emails, tâches, ...~;
                \item Responsabilité du compte~: Salariés (techniciens, commerciaux, ...) responsables de ce compte~;
                \item Données de contact~: Informations de contact du client.
            \end{itemize}

        \item[Gestion des commandes~:] ~ \\
            \begin{itemize}
                \item Objectifs des ventes~: Comparaison des objectifs de vente avec les commande entrantes, les opportunités et une projection sur la base des commandes réalisées~;
                \item Historique du volume de commandes~: Volume de commandes à une date donnée~;
                \item Statistiques de contrats~: Nombre de contrats~: actifs, signés récemment, expirés, ...~;
                \item Taux de reconduction de contrats~;
                \item Taux de contrats dénoncés~;
                \item Retour d'expérience client~: Vue sur le taux de satisfaction des clients.
            \end{itemize}

        \item[Facturation~:] ~ \\
            \begin{itemize}
                \item Factures en cours / non validées~:
                \item Volume facturé~;
                \item Services commandés, confirmés et facturés~;
            \end{itemize}

        \item[Services~:] ~ \\
            \begin{itemize}
                \item Base de connaissances~: articles de la base de connaissance de SPIE, incidents les plus fréquemment gérés par le pôle maintenance, ...;
                \item Délai moyen des demandes de services~: délai moyen pour terminer des demandes de service (indique l'efficacité des procédures internes et permet dese concentrer sur les éléments de procédures à améliorer);
                \item Services commandés, confirmés et facturés~;
                \item Demandes de services en cours~;
                \item Taux de résolution au premier appel~: taux de demandes client résolues au cours de la première intervention~;
                \item Taux de traitement des demandes de services~: vue hebdomadaire des demandes entrantes / traitées.
            \end{itemize}

    \end{description}
