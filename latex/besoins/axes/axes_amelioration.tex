\chapter{Attentes client}

	Nous avons été contactés par la société SPIE Sud-Est afin d'améliorer certains points dans le processus de gestion des contrats de maintenance. Il s'agit de proposer des solutions pour :

    \begin{itemize}
        \item développement de procédures métier et de supports d'exploitation par les entités de maintenance et service
        \item standardisation des procédures métier et de supports d'exploitation pour les entités exerçant le même métier sur le même secteur d'activité client
        \item analyse des risques propres à chaque métier sur le même secteur d'activité client
        \item amélioration de la définition des limites des interfaces avec les autres processus
        \item mise en place d'un Info centre dur l'intranet
    \end{itemize}

    Pour cela, nous proposons des interventions sur différents niveaux :

\section{L'organisation}
    Concernant la partie organisationnelle de la maintenance, il est important que les meilleurs techniciens soient mobilisés après la signature du contrat entre les deux parties. En effet, nous pourrions imaginer un système permettant de recommander certains techniciens en fonction de leur aptitude à résoudre tel ou tel type de problème. Les techniciens interviendraient alors beaucoup plus efficacement, en restant dans leur domaine de compétences.

    On pourrait également imaginer qu'un même technicien continue d'être affecté au même contrat, lui permettant de suivre l'évolution et d'être efficace sans perdre de temps en communication à devoir sans cesse ré-expliquer le problème à un autre technicien.

    Il est également nécessaire de porter une attention toute particulière à la documentation du travail des techniciens (pour faciliter le changement de technicien en cas de besoin).

    On peut également engager une nouvelle ressource, chargée d'analyser, pour chaque projet, les causes de la réussite ou de l'échec. Cela permettrait de tirer des enseignements et d'accumuler de l'expérience.


\section{Les indicateurs}

    Nous avons dégagés des indicateurs permettant de suivre de manière quantitative la qualité et l'efficacité des processus de SPIE Sud-Est.

    \begin{description}
        \item[Gestion des comptes~:] ~ \\
            \begin{itemize}
                \item Détails du compte~: Synthèse des données principales du compte (id, adresse, données bancaires, ...)~;
                \item Historique d'activité du compte~: Nombre de rendez-vous, appels téléphoniques, emails, tâches, ...~;
                \item Responsabilité du compte~: Salariés (techniciens, commerciaux, ...) responsables de ce compte~;
                \item Données de contact~: Informations de contact du client.
            \end{itemize}

        \item[Gestion des commandes~:] ~ \\
            \begin{itemize}
                \item Objectifs des ventes~: Comparaison des objectifs de vente avec les commande entrantes, les opportunités et une projection sur la base des commandes réalisées~;
                \item Historique du volume de commandes~: Volume de commandes à une date donnée~;
                \item Statistiques de contrats~: Nombre de contrats~: actifs, signés récemment, expirés, ...~;
                \item Taux de reconduction de contrats~;
                \item Taux de contrats dénoncés~;
                \item Retour d'expérience client~: Vue sur le taux de satisfaction des clients.
            \end{itemize}

        \item[Facturation~:] ~ \\
            \begin{itemize}
                \item Factures en cours / non validées~:
                \item Volume facturé~;
                \item Services commandés, confirmés et facturés~;
            \end{itemize}

        \item[Services~:] ~ \\
            \begin{itemize}
                \item Base de connaissances~: articles de la base de connaissance de SPIE, incidents les plus fréquemment gérés par le pôle maintenance, ...;
                \item Délai moyen des demandes de services~: délai moyen pour terminer des demandes de service (indique l'efficacité des procédures internes et permet de se concentrer sur les éléments de procédures à améliorer);
                \item Services commandés, confirmés et facturés~;
                \item Demandes de services en cours~;
                \item Taux de résolution au premier appel~: taux de demandes client résolues au cours de la première intervention~;
                \item Taux de traitement des demandes de services~: vue hebdomadaire des demandes entrantes / traitées.
            \end{itemize}

    \end{description}

\section{Les procédures}

\begin{description}
    \item[Retour d'experience client~:] Un procssus ajouté dans le sous-processus Solde de l'affaire et du contrat pour les retours client afin de pouvoir \textit{brainstormer} ensuite sur des solutions aux problèmes ayant
    pu survenir dans le bilan.
    \item[R\'eunion retour d'experience interne~:] Nouveau processus dans le sous-processus Solde de l'affaire et du contrat pour pouvoir r\'ecup\'erer les retours des differents acteurs du contrat toujours pour preparer
    le bilan du contrat.
    \item[Bilan du contrat~:] Nouveau processus, dans le sous processus Solde de l'affaire et du contrat egalement, afin de prendre d'éventuelles d\'ecisions \`a partir des retours.
    \item[Revues des processus~:] est un tout nouveau processus parrall\`ele \`a ceux \'existant pour les superviser.
\end{description}


\section{Des nouvelles technologies}

    L'intégration de nouvelles technologies peut permettre à SPIE de se démarquer de la concurrence en changeant de manière radicale la gestion de certaines opérations.

    Nous allons nous intéresser à certaines technologies que SPIE pourrait intégrer~:


    \subsubsection{Gestion de la Relation Client avec un CRM}

        Les outils CRM (\textit{Customer Relationship Management}, ou Gestion de la Relation Client) permettent une gestion optimale la Relation Client avec des outils de modélisation, de \textit{reporting} et de prédiction. Ils pourraient donc remplacer les multiples applications utilisées par SPIE pour gérer les comptes clients, les appels, ...

    \subsubsection{SI utilisable pendant les interventions (application mobile)}

        Les techniciens ne peuvent actuellement pas utiliser les applications SPIE pendant une intervention: ils ne disposent pas de tablettes fournies à ce but et les applications SPIE ne sont pas optimisés (ou même parfois utilisables) sur des terminaux mobiles.

    \subsubsection{Moteur de recommandation et de préventions des risques}

        SPIE peut valoriser ses années d'expérience dans le domaine de la maintenance en créant un moteur de recommandation exploitant toutes les données récoltées lors d'interventions passées afin de proposer aux techniciens des recommandations similaires à celle qu'ils souhaitent effectuer.
