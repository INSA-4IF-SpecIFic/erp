\chapter{Attentes client}

\chapter{Am\'elioration des Processus}

Apr\`es analyse du processus m\'etier fournis par SPIE, on remarque que la Direction General (DG) n'intervient
pas du tout dans le processus gestion de contrat et de maintenance et services. C'est l\`a alors un probleme
qui devrait \^etre rectifier a l'avenir. En effet, il est vitale que les acteurs intervenants dans ses processus,
quandrant ou non, soient eux aussi cadr\'e. Par Ailleurs, cela permetrait de solidariser l'ensemble des contrats de SPIE, permetant de repartir les ressources
physique et humaine de mani\`eres plus egales \`a travers ces derniers. Cela impliquerait alors la cr\'eation d'un processus
de revues de l'ensembles des autres processus en cours d'execution, ou termin\'es.

Un dernier probl\'eme r\'ecurant dans les processus actuels, est qu'il n'y aucune \'etape d\'edi\'ee au retour, que
se soit du client, ou en interne. Pour connaitre les cause et cons\'equances de chaque faiblaisses aillant \'et\'e survenue
et les rectifiers, nous rajouterions alors deux \'etape dans le sous processus Solde de l'Affaire et du Contrat~:

\begin{itemize}
    \item Un pour les retours client \`a fin de pouvoir brain-stormer ensuite sur des solutions aux probl\'emes aillant
    put survenir~;
    \item L'autre pour les internes des differents acteurs. La Direction g\'enerale aurait ici sa place en cas
    de d'\'ecision de force majeur (comme par exemple le renvoi d'un employer aillant fait une faute professionel).
\end{itemize}
