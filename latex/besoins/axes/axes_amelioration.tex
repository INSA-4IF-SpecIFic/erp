\chapter{Attentes client}
	Nous avons été contactés par la société SPIE Sud-Est afin d'améliorer certains points dans le processus de gestion des contrats de maintenance. Il s'agit de proposer des solutions pour :

	\begin{itemize}
        \item développement de procédures métier et de supports d'exploitation par les entités de maintenance et service
        \item standardisation des procédures métier et de supports d'exploitation pour les entités exerçant le même métier sur le même secteur d'activité client
        \item analyse des risques propres à chaque métier sur le même secteur d'activité client
        \item amélioration de la définition des limites des interfaces avec les autres processus
        \item mise en place d'un Info centre dur l'intranet
    \end{itemize}

    Pour cela, nous proposons des interventions sur différents niveaux :

    \section{Les procédures}

    \section{Des nouvelles technologies}

    \section{L'organisation}

    \section{Les indicateurs}

        Nous avons dégagés des indicateurs permettant de suivre de manière quantitative la qualité et l'efficacité des processus de SPIE Sud-Est.

        \begin{description}
            \item[Gestion des comptes~:] ~ \\
                \begin{itemize}
                    \item Détails du compte~: Synthèse des données principales du compte (id, adresse, données bancaires, ...)~;
                    \item Historique d'activité du compte~: Nombre de rendez-vous, appels téléphoniques, emails, tâches, ...~;
                    \item Responsabilité du compte~: Salariés (techniciens, commerciaux, ...) responsables de ce compte~;
                    \item Données de contact~: Informations de contact du client.
                \end{itemize}

            \item[Gestion des commandes~:] ~ \\
                \begin{itemize}
                    \item Objectifs des ventes~: Comparaison des objectifs de vente avec les commande entrantes, les opportunités et une projection sur la base des commandes réalisées~;
                    \item Historique du volume de commandes~: Volume de commandes à une date donnée~;
                    \item Statistiques de contrats~: Nombre de contrats~: actifs, signés récemment, expirés, ...~;
                    \item Taux de reconduction de contrats~;
                    \item Taux de contrats dénoncés~;
                    \item Retour d'expérience client~: Vue sur le taux de satisfaction des clients.
                \end{itemize}

            \item[Facturation~:] ~ \\
                \begin{itemize}
                    \item Factures en cours / non validées~:
                \end{itemize}

        \end{description}



        % Facturation client

        %  * Synthèse du projet client

        % Fournit des indicateurs liés au projet du client, comprenant les quantités et les valeurs nettes, à partir de commandes clients et de factures.

        % * Factures en cours et non validées

        % Affiche les détails des factures en cours et non validées.

        % * Volume facturé

        % Affiche le volume et la quantité des factures, avoirs, et factures de correction avec le statut "Validé" ou "Validation annulée".

        % * Services - commandé, confirmé, facturé

        % Affiche les quantités et les valeurs des postes de produits services commandés, confirmés et facturés dans les commandes clients. Permet d'évaluer la situation actuelle de ces postes.

        % Centre de services

        % * Base de connaissances

        % Affiche les articles de la base de connaissances les plus couramment affectés aux demandes et aux ordres de service. Identifie les incidents les plus fréquemment gérés par votre centre de services.

        % * Délai moyen des demandes de services

        % Affiche le délai moyen nécessaire pour terminer les demandes de service. Indique l'efficacité des procédures internes et vous permet de vous concentrer sur les éléments de procédures à améliorer.

        % * Demandes de services en portefeuille

        % Affiche le nombre de demandes de service en suspens.

        % * Respect de l'échéane

        % Affiche le taux de demandes de service satisfaites dans les délais d’intervention et d’exécution planifiés issus de l’échéance du Service Level Objective.

        % * Taux de résolution au premier appel

        % Le taux de résolution au premier appel indique le taux de demandes client résolues au cours de la première itération avec le client.

        % * Taux de traitements de demande de services

        % Compare vos demandes de service entrantes et terminées pour les sept derniers jours.
