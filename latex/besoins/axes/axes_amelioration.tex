\chapter{Attentes client}

Nous avons été contactés par la société SPIE Sud-Est afin d'améliorer certains points dans le processus de gestion des contrats de maintenance. Il s'agit de proposer des solutions pour :

\begin{itemize}
    \item développement de procédures métier et de supports d'exploitation par les entités de maintenance et service
    \item standardisation des procédures métier et de supports d'exploitation pour les entités exerçant le même métier sur le même secteur d'activité client
    \item analyse des risques propres à chaque métier sur le même secteur d'activité client
    \item amélioration de la définition des limites des interfaces avec les autres processus
    \item mise en place d'un Info centre dur l'intranet
\end{itemize}

Pour cela, nous proposons des interventions sur différents niveaux :


\section{Les procédures}

Apr\`es analyse du processus m\'etier fournis par SPIE, on remarque que la Direction General (DG) n'intervient
pas du tout dans le processus gestion de contrat et de maintenance et services. C'est l\`a alors un probleme
qui devrait \^etre rectifier a l'avenir. En effet, il est vitale que les acteurs intervenants dans ses processus,
quandrant ou non, soient eux aussi cadr\'e. Par Ailleurs, cela permetrait de solidariser l'ensemble des contrats de SPIE, permetant de repartir les ressources
physique et humaine de mani\`eres plus egales \`a travers ces derniers. Cela impliquerait alors la cr\'eation d'un processus
de revues de l'ensembles des autres processus en cours d'execution, ou termin\'es.

Un dernier probl\'eme r\'ecurant dans les processus actuels, est qu'il n'y aucune \'etape d\'edi\'ee au retour, que
se soit du client, ou en interne. Pour connaitre les cause et cons\'equances de chaque faiblaisses aillant \'et\'e survenue
et les rectifiers, nous rajouterions alors deux \'etape dans le sous processus Solde de l'Affaire et du Contrat~:

\begin{itemize}
    \item Un pour les retours client \`a fin de pouvoir brain-stormer ensuite sur des solutions aux probl\'emes aillant
    put survenir~;
    \item L'autre pour les internes des differents acteurs. La Direction g\'enerale aurait ici sa place en cas
    de d'\'ecision de force majeur (comme par exemple le renvoi d'un employer aillant fait une faute professionel).
\end{itemize}


\section{Des nouvelles technologies}

L'intégration de nouvelles technologies peut permettre à SPIE de se démarquer de la concurrence en changeant de manière radicale la gestion de certaines opérations.

Nous allons nous intéresser à certaines technologies que SPIE pourrait intégrer~:


    \subsubsection{Gestion de la Relation Client avec un CRM}

        Les outils CRM (\textit{Customer Relationship Management}, ou Gestion de la Relation Client) permettent une gestion optimale la Relation Client avec des outils de modélisation, de reporting et de prédiction.

        L'utilisation d'outils dédiés permet autant d'améliorer les processus de Relation Client que de prédire les attentes de clients et ainsi prévenir les échecs et mieux s'y préparer. Là où SPIE utilise actuellement plusieurs outils (\textit{Clarify}, \textit{ADV}, \textit{SUPRA}) aux infrastructures parfois séparées, l'utilisation d'un seul CRM simplifierai les processus de facturation, de suivi contrats et autres processus clients, et permettrait une meilleure intégration entre ces derniers.

    \subsubsection{SI utilisable pendant les interventions (application mobile)}

        Les techniciens ne peuvent actuellement pas utiliser les applications SPIE pendant une intervention: ils ne disposent pas de tablettes fournies à ce but et les applications SPIE ne sont pas optimisés (ou même parfois utilisables) sur des terminaux mobiles.

    \subsubsection{Moteur de recommandation et de préventions des risques}

        SPIE peut valoriser ses années d'expérience dans le domaine de la maintenance en créant un moteur de recommandation exploitant toutes les données récoltées lors d'interventions passées afin de proposer aux techniciens des recommandations similaires à celle qu'ils souhaitent effectuer.

\section{L'organisation}

\section{Les indicateurs}

