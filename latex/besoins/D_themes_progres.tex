\chapter{Thèmes de progrès}

\section{Thèmes de progrès fonctionnels}

\section{Thèmes de progrès organisationnels}

Concernant la partie organisationnelle de la maintenance, il est important que les meilleurs techniciens soient mobilisés après la signature du contrat entre les deux parties. En effet, nous pourrions imaginer un système permettant de recommander certains techniciens en fonction de leur aptitude à résoudre tel ou tel type de problème. Les techniciens interviendraient alors beaucoup plus efficacement, en restant dans leur domaine de compétences.

On pourrait également imaginer qu'un même technicien continue d'être affecté au même contrat, lui permettant de suivre l'évolution et d'être efficace sans perdre de temps en communication à devoir sans cesse ré-expliquer le problème à un autre technicien.

Il est également nécessaire de porter une attention toute particulière à la documentation du travail des techniciens (pour faciliter le changement de technicien en cas de besoin).

On peut également engager une nouvelle ressource, chargée d'analyser, pour chaque projet, les causes de la réussite ou de l'échec. Cela permettrait de tirer des enseignements et d'accumuler de l'expérience.

Le Directeur Général n'est pas assez impliqué dans le processus de maintenance. On remarque qu'il n'intervient dans aucun de ses sous-processus. Il serait bon de le faire intervenir.
Pour les gros contrats, le Directeur Général peut intervenir en amont du sous-processus «~Offre et revue d'offre~», lors de l'«~Opportunité de contrat de service~». Il peut participer à une discussion initiale par téléphone aux gros clients potentiels, pour leur présenter le service qui pourrait leur être offert.

Le Directeur Général pourrait également superviser de temps en temps l'équipe, en intervenant dans le processus «~Revue des processus~» (soldés et en cours). Cela lui permettrait d'indiquer à l'équipe le bon cap à suivre.

\section{Thèmes de progrès technologiques}

    L'intégration de nouvelles technologies peut permettre à SPIE de se démarquer de la concurrence en changeant de manière radicale la gestion de certaines opérations.

    Nous allons nous intéresser à certaines technologies que SPIE pourrait intégrer~:

        \subsubsection{Gestion de la Relation Client avec un CRM}

            Quand on s'intéresse à la cartographie générale du SI de SPIE, la première chose qu'on remarque est l'absence d'une gestion de la Relation Client en utilisant un outil dédié, de type CRM.

            Les outils CRM (\textit{Customer Relationship Management}, ou Gestion de la Relation Client) permettent une gestion optimale la Relation Client avec des outils de modélisation, de reporting et de prédiction.

            L'utilisation d'outils dédiés permet autant d'améliorer les processus de Relation Client que de prédire les attentes de clients et ainsi prévenir les échecs et mieux s'y préparer. Là où SPIE utilise actuellement plusieurs outils (\textit{Clarify}, \textit{ADV}, \textit{SUPRA}) aux infrastructures parfois séparées, l'utilisation d'un seul CRM simplifierai les processus de facturation, de suivi contrats et autres processus clients, et permettrait une meilleure intégration entre ces derniers.

            Les outils CRM permettent en outre de mener des campagnes de prospection, de suivre les opportunités commerciales et de fidéliser les clients.

            On peut citer à titre d'exemple \textit{Salesforce}, l'un des leaders du marché qui propose un CRM en mode \textit{SaaS}.

        \subsubsection{SI utilisable pendant les interventions (application mobile)}

            Les techniciens ne peuvent actuellement pas utiliser les applications SPIE pendant une intervention: ils ne disposent pas de tablettes fournies à ce but et les applications SPIE ne sont pas optimisés (ou même parfois utilisables) sur des terminaux mobiles.

            Fournir des tablettes aux techniciens et leur donner accès au système d'information de SPIE leur permettrait de proposer une meilleur service aux clients chez lesquels ils interviennent en ayant constamment sous les yeux les informations relatives à des interventions précédentes (par eux ou d'autres techniciens), et pouvant entrer au fur et à mesure des informations sur le déroulement de l'intervention afin d'assurer une meilleure réactivité et afin qu'aucune information ne soit oubliée (ce qui risque d'arriver si le technicien ne fait l'entrée de données qu'une fois l'intervention terminée).

            Il faut noter que les éditeurs d'ERPs, et plus généralement de logiciels modernes adoptent souvent aujourd'hui une politique de \textit{``Mobile first''} et proposent ainsi des outils accessibles aussi bien depuis ordinateur que depuis mobile et tablettes, sans sacrifier les fonctionnalités de l'ERP ou son ergonomie. SAP ByDesign s'adapte ainsi directement au terminal utilisé pour y accéder, et l'éditeur de CRM \textit{Salesforce} propose un \textit{framework} CRM adapté pour mobile, \textit{Salesforce1}, s'intégrant avec la totalité de sa ligne de produit.


        \subsubsection{Moteur de recommandation et de préventions des risques}

            SPIE peut valoriser ses années d'expérience dans le domaine de la maintenance en créant un moteur de recommandation exploitant toutes les données récoltées lors d'interventions passées afin de proposer aux techniciens des recommandations similaires à celle qu'ils souhaitent effectuer.

            De cette manière, l'expérience des techniciens SPIE sera réellement mutualisée et mise à profit~:

            \begin{itemize}
                \item Priorisation des points à vérifier pendant l'intervention, ce qui permet d'assurer des interventions rapides et efficaces~;
                \item Prévention d'erreurs commises lors d'interventions similaires~;
                \item Possibilité de contacter facilement un technicien étant intervenu dans un cas similaire afin de faire un partage d'information.
            \end{itemize}

            Les critères à privilégier pour recommander des livraisons sont nombreux. En voici quelque uns~:
            \begin{itemize}
                \item Client visé par l'intervention~;
                \item Matériel utilisé~;
                \item Zone géographique~;
                \item ``Symptomes'' de panne~;
                \item ...
            \end{itemize}

\section{Indicateurs de performance}

    Les indicateurs de performance servent à évaluer la performance des processus de l'entreprise, aussi bien dans le cas d'une PME comme SPIE Sud-Est que dans celui d'un grand groupe. Ils sont mis en place en fonction du type d'activité et il peut s'agir d'indicateur de production comme d'indicateurs de logistique ou encore de performance sociale. Ces indicateurs sont quantitatifs et sont centralisés dans un tableau de bord.

    Nous avons donc dégagés les indicateurs permettant de suivre de manière quantitative la qualité et l'efficacité des processus de SPIE Sud-Est~:

    \begin{description}
        \item[Gestion des comptes~:] ~ \\
            \begin{itemize}
                \item Détails du compte~: Synthèse des données principales du compte (id, adresse, données bancaires, ...).~;
                \item Historique d'activité du compte~: Nombre de rendez-vous, appels téléphoniques, emails, tâches, ...~;
                \item Responsabilité du compte~: Salariés (techniciens, commerciaux, ...) responsables de ce compte~;
                \item Données de contact~: Informations de contact du client.
            \end{itemize}

        \item[Gestion des commandes~:] ~ \\
            \begin{itemize}
                \item Objectifs des ventes~: Comparaison des objectifs de vente avec les commande entrantes, les opportunités et une projection sur la base des commandes réalisées~;
                \item Historique du volume de commandes~: Volume de commandes à une date donnée~;
                \item Statistiques de contrats~: Nombre de contrats~: actifs, signés récemment, expirés, ...~;
                \item Taux de reconduction de contrats~;
                \item Taux de contrats dénoncés~;
                \item Retour d'expérience client~: Vue sur le taux de satisfaction des clients.
            \end{itemize}

        \item[Facturation~:] ~ \\
            \begin{itemize}
                \item Factures en cours / non validées~: Détails des factures en cours et non validées~;
                \item Volume facturé~: Volume et nombre de factures validées~;
                \item Services commandés, confirmés et facturés~: Quantité des postes de produits services commandés, confirmés et facturés dans les commandes clients. Permet d'évaluer la situation actuelle de ces postes.~;
            \end{itemize}

        \item[Services~:] ~ \\
            \begin{itemize}
                \item Base de connaissances~: articles de la base de connaissance de SPIE, incidents les plus fréquemment gérés par le pôle maintenance, ...;
                \item Délai moyen des demandes de services~: délai moyen pour terminer des demandes de service (indique l'efficacité des procédures internes et permet de se concentrer sur les éléments de procédures à améliorer);
                \item Services commandés, confirmés et facturés~;
                \item Demandes d'interventions en cours: Volume des demandes d'intervention en cours de réalisation. Un volume de demandes d'interventions en cours trop haut indique que l'entreprise est surchargée et doit allouer plus de moyens et de personnel afin de traiter ces demandes~;
                \item Taux de résolution au premier appel~: taux de demandes client résolues au cours de la première intervention~;
                \item Taux de traitement des demandes de services~: taux des demandes traitées sur une durée donnée (hebdomadaire en général). Un taux élevé indique une grande réactivité et une certaine efficacité dans le traitement des demandes, un taux faible peut produire plus d'attente aux niveaux des clients et donc une moins bonne qualité de service.
            \end{itemize}

    \end{description}
