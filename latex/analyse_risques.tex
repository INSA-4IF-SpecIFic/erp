\chapter*{Analyse des risques}
\addcontentsline{toc}{chapter}{Analyse des risques}
\chaptermark{Analyse des risques}

\subsection*{Méthodologie}
\addcontentsline{toc}{subsection}{Méthodologie}

Nous évaluons les risques de ce projet selon leur gravité et leur probablité. Les risques improbables et/ou à gravité mineure sont ainsi plus acceptables que ceux cumulant de fortes probabilités et gravités.

Les classes de probabilités que nous utilisons sont les suivantes:

\begin{itemize}
  \item \textbf{1}: Improbable 
  \item \textbf{2}: Rare
  \item \textbf{3}: Moyennement probable
  \item \textbf{4}: Probable
  \item \textbf{5}: Très probable
\end{itemize}

De même, les classes de gravité sont:

\begin{itemize}
  \item \textbf{1}: Mineure 
  \item \textbf{2}: Moyenne
  \item \textbf{3}: Majeure
  \item \textbf{4}: Critique
  \item \textbf{5}: Catastrophique
\end{itemize}

Nous choisissons donc les risques innacceptables (nécessitant une action) selon la matrice suivante:

\begin{tabular}{|l|l|l|l|l|l|}
     \hline
         P/G & \textbf{1} & \textbf{2} & \textbf{3} & \textbf{4} & \textbf{5}  \\ \hline
         \textbf{1} &   &   &   &   & X \\ \hline
         \textbf{2} &   &   &   & X & X \\ \hline
         \textbf{3} &   &   & X & X & X \\ \hline
         \textbf{4} &   & X & X & X & X \\ \hline
         \textbf{5} & X & X & X & X & X \\
     \hline
\end{tabular}

\subsection*{Risques}
\addcontentsline{toc}{subsection}{Risques}

À partir de ces critères là, nous avons identifiés les risques suivants:

\begin{tabular}{|l||l|l|l|}
   \hline	
	   \textbf{Risque} & \textbf{Type} & \textbf{Probabilité} & \textbf{Gravité} \\ \hline \hline
	   Dépassement des échéances         & Organisation & 4 & 3  \\ \hline
	   Incohérence des livrables         & Organisation & 3 & 3  \\ \hline
  	   Absence d'un membre de l'équipe   & Organisation & 2 & 4  \\ \hline
	   Non respect des exigences         & Production   & 3 & 4  \\ \hline
	   Solution inadaptée                & Production   & 3 & 4  \\ \hline
	   Panne des outils de collaboration & Matériel     & 2 & 4  \\
   \hline
\end{tabular}

\subsection*{Plan d'action}
\addcontentsline{toc}{subsection}{Plan d'action}

Deux types d'actions peuvent être entreprises pour gérer ces risques:

\begin{itemize}
  \item \textbf{Actions de prévention}: diminuent la probabilité 
  \item \textbf{Actions de protection}: diminuent la gravité 
\end{itemize}

Voici donc le plan d'action pour gérer les risques cités précédemment:

% à faire 