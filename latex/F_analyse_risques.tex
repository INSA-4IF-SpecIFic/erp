\chapter{Analyse des risques}

\section{Méthodologie}
Nous évaluons les risques de ce projet selon leur gravité et leur probablité. Les risques improbables et/ou à gravité mineure sont ainsi plus acceptables que ceux cumulant de fortes probabilités et gravités.

Les classes de probabilités que nous utilisons sont les suivantes:

\begin{itemize}
  \item \textbf{1} : Improbable 
  \item \textbf{2} : Rare
  \item \textbf{3} : Moyennement probable
  \item \textbf{4} : Probable
  \item \textbf{5} : Très probable
\end{itemize}

De même, les classes de gravité sont:

\begin{itemize}
  \item \textbf{1} : Mineure 
  \item \textbf{2} : Moyenne
  \item \textbf{3} : Majeure
  \item \textbf{4} : Critique
  \item \textbf{5} : Catastrophique
\end{itemize}

Nous choisissons donc les risques innacceptables (nécessitant une action) selon la matrice suivante:

\begin{tabular}{|c|l|l|l|l|l|}
     \hline
         P/G & \textbf{1} & \textbf{2} & \textbf{3} & \textbf{4} & \textbf{5}  \\ \hline
         \textbf{1} &   &   &   &   & x \\ \hline
         \textbf{2} &   &   &   & x & x \\ \hline
         \textbf{3} &   &   & x & x & x \\ \hline
         \textbf{4} &   & x & x & x & x \\ \hline
         \textbf{5} & x & x & x & x & x \\
     \hline
\end{tabular}

\newpage

\section{Risques}
À partir de ces critères là, nous avons identifié les risques suivants:

\begin{tabular}{|l||c|c|c|}
   \hline	
	   \textbf{Risque} & \textbf{Type} & \textbf{Probabilité} & \textbf{Gravité} \\ \hline \hline
	   Dépassement des échéances         & Organisation & 4 & 3  \\ \hline
	   Incohérence des livrables         & Organisation & 3 & 3  \\ \hline
  	   Absence d'un membre de l'équipe   & Organisation & 2 & 4  \\ \hline
	   Non respect des exigences         & Production   & 3 & 4  \\ \hline
	   Solution inadaptée                & Production   & 3 & 4  \\ \hline
	   Panne des outils de collaboration & Matériel     & 2 & 4  \\
   \hline
\end{tabular}

\section{Plan d'action}

Deux types d'actions peuvent être entreprises pour gérer ces risques:

\begin{itemize}
  \item \textbf{Actions de prévention}: diminuent la probabilité du risque
  \item \textbf{Actions de protection}: diminuent la gravité du risque
\end{itemize}

Voici donc notre plan d'action pour gérer les risques précédemment cités:

\begin{itemize}

  \item \textbf{Dépassement des échéances}
        \begin{itemize}
          \item \textbf{Prévention}: Établir un phasage précis, et une distribution équitable des tâches.
          \item \textbf{Protection}: Séparer les tâches en parties indépendantes pour ne pas ralentir toute l'équipe à cause du retard d'un des membres.
        \end{itemize}

  \item \textbf{Incohérence des livrables}
        \begin{itemize}
          \item \textbf{Prévention}: Effectuer un suivi qualité continu des livrables et réaliser des réunions pour faire une synthèse du travail de l'équipe.
        \end{itemize}

  \item \textbf{Absence d'un membre de l'équipe}
        \begin{itemize}
          \item \textbf{Prévention}: Dans le cas d'un imprévu (maladie, ...), les heures de travail râtées peuvent être récupérées en hors-horaire.
          \item \textbf{Protection}: Comme pour le premier risque, séparer les tâches et responsabilités en parties indépendantes (au possible).
        \end{itemize}


  \item \textbf{Non respect des exigences}
        \begin{itemize}
          \item \textbf{Prévention}: Établir une liste d'exigences précise à travers une étude métier détaillée et un recueuil avancé des besoins du client (avec - si ambiguïté - des réunions pour préciser ces derniers). Suivi de la complétion de ces exigeances tout au long du projet.
        \end{itemize}


  \item \textbf{Solution inadaptée}
        \begin{itemize}
          \item \textbf{Prévention}:  Assister à la formation ERP pour mieux appréhender ces solutions. Réaliser un recueuil détaillé des besoins client (voir risque précédent). Étudier et rechercher les solutions existantes les mieux adaptées à ces besoins.
        \end{itemize}


  \item \textbf{Panne des outils de collaboration}
        \begin{itemize}
          \item \textbf{Prévention}: Choisir des outils dont l'infrastructure est robuste (haute disponibilité des serveurs).
          \item \textbf{Protection}: Constamment garder une copie locale du travail de l'équipe, en prévision d'indisponibilité des outils.
        \end{itemize}

\end{itemize}
