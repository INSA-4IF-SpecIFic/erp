\chapter{Organisation de l'équipe}
Cette partie a pour but de présenter les différents rôles de l'équipe projet que nous avons constituée. Pour chacun des rôles, nous présenterons leurs attributions.

\section{Chef de projet~: Valentina~\textsc{Zantedeschi}}
Chargée de la planification et de l'organisation, elle est également garante de la coordination et de la cohésion du groupe.

Elle effectuera le \textbf{suivi stratégique} du projet en évaluant les risques et en s'assurant du respect des objectifs dans les délais impartis. Elle veillera au bon \textbf{pilotage opérationnel} en planifiant les tâches et en les encadrant. Elle mettra tout en œuvre pour garantir une \textbf{organisation humaine} efficace en définissant le rôle de chacun des membres, en leur attribuant des tâches et en résolvant les éventuels conflits. Enfin, elle s'assurera du bon \textbf{pilotage de la production}, en suivant les résultats et les livrables.


\section{Responsable qualité~: Marina~\textsc{Julien}}
D'une manière générale, elle veillera à la \textbf{qualité des livrables} fournis au client. Elle s'assurera également de la cohésion entre les livrables, de la maintenance et du \textbf{respect du Plan d'Assurance Qualité}.

\section{Expert méthode et outils~: Thierry \textsc{Cantenot}}
Il sera chargé de \textbf{conseiller}, d'\textbf{assister}, de \textbf{former} et d'\textbf{informer}. Il effectuera également la \textbf{veille technologique} quant aux outils utilisés et \textbf{proposera des évolutions} qu'il juge nécessaires.


\section{Responsable communication~: Martin~\textsc{Wetterwald}}
Canal de communication principal, il sera l'interface privilégiée entre l'équipe et les clients, en plus de garantir une bonne communication intra-équipe. Il sera le premier à être contacté par les clients en cas de problème avec les livrables (difficulté de transmission ou non-validation de la part du client).


\section{Expert ERP~: Ahmed~:\textsc{Kachkach}}
