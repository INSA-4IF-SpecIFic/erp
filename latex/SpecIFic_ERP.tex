
% --------------------------------------------------------------- CONFIGURATIONS

%ifdef TWOSIDE
	%\documentclass[a4paper,12pt,final,twoside,openright]{book}
%elif ONESIDE
	\documentclass[a4paper,12pt,final,oneside]{book}
%endif

\usepackage{rapport}


% -------------------------------------------------------------- META: CONSTANTS

\newcommand{\reporttitle}{ERP}
\newcommand{\enseignants}{Youssef~\textsc{Amghar}\\ Anne~\textsc{Legait}\\ Pierre-Alain~\textsc{Millet}\\ Mohamed~\textsc{Ouhalima}}
\newcommand{\reportauthor}{Guillaume~\textsc{Abadie}\\ Thierry~\textsc{Cantenot}\\ Marina~\textsc{Julien}\\ Ahmed~\textsc{Kachkach}\\ Martin~\textsc{Wetterwald}\\ Valentina~\textsc{Zantedeschi}}
\newcommand{\hexanome}{SpecIFic}
\newcommand{\reportsubject}{Livrable de projet}
\newcommand{\stagetopic}{Blablabla test}
\newcommand{\dateperiod}{du 17 décembre 2013 au 11 mars 2014}
\newcommand{\HRule}{\rule{\linewidth}{0.5mm}}
\setlength{\parskip}{1ex} % Espace entre les paragraphes

\hypersetup{
	pdftitle={\reporttitle},%
		pdfauthor={\reportauthor},%
		pdfsubject={\reportsubject},%
		pdfkeywords={INSA Lyon}
}

\title{\reporttitle}
\author{\reportauthor}
%\setcounter{tocdepth}{4}


% ------------------------------------------------------------------------- FILE

\begin{document}


    % ------------------------------------------------------------------- HEADER

	\renewcommand{\chaptername}{} %\renewcommand{\thechapter}{}
	\renewcommand{\contentsname}{Sommaire}

	\pagestyle{empty}
	\pagenumbering{Roman}


    % ------------------------------------------------------------ HEADER: TITLE

	\include{title}

	%ifdef TWOSIDE
		%\cleardoublepage
	%endif

	%\include{title2}

	%ifdef TWOSIDE
		%  \newpage
		%	\null
		%	\vfill
	%endif


    % --------------------------------------------------- HEADER: CONFIGURATIONS

	\sloppy          % Justification moins stricte : des mots ne dépasseront pas des paragraphes

    \frontmatter
		\pagestyle{empty}
		\tableofcontents
		\addtocontents{toc}{\protect\thispagestyle{empty}}

	\mainmatter
	\pagestyle{headings}

	\renewcommand{\thechapter}{\Alph{chapter}}
	\renewcommand{\chaptermark}[1]{\markboth{\MakeUppercase{\chaptername\ \thechapter.\ #1}}{}}
	\renewcommand{\sectionmark}[1]{\markright{\thesection{} #1}}


    % ------------------------------------------------------------------ CONTENT

	\chapter{Introduction}

\section{Objet du projet}

Le but du projet est la réalisation de l’étude préalable préliminaire de la conception et de
l’automatisation du système d’information du domaine gestion des contrats de maintenance, chez SPIE SUD EST.

L'objectif du projet s'étend donc sur plusieurs points~:
\begin{itemize}
    \item Spécifications de solutions informatiques (standard, spécifique)~;
    \item Mise en oeuvre d’outils et de méthode de conception~;
    \item Modélisation de la solution (élaboration de scénarii)~;
    \item Choix des moyens, évaluation, etc~;
    \item Elaboration du plan qualité.
\end{itemize}

\paragraph{Note Importante} Le but du projet étant la réalisation d'une étude préalable, nous nous
limiterons aux phases de spécifications et de conception du système d’information. En d'autres termes,
nous ne prendrons donc pas en charge les phases suivantes dans l’étude préalable, c’est à dire~:
l’étude détaillée ainsi que la réalisation.


\section{Contexte général du projet}

\subsection{SPIE}

SPIE est une multinationale d'origine francaise, et réalisant encore aujourd'hui 62\% de son
chiffre d'affaire en France, 25\% en Europe et 13\% hors Europe. Son objectif est d'améliorer le confort
de tous les jours via divers systèmes, en réalisant 66\% de son chiffre d'affaire par la maintenance de ceux-ci.

SPIE est composé des services~:

\begin{itemize}
    \item Direction Générale (DG)~;
    \item Direction des Ressources Humaines (DRH)~;
    \item Direction Administrative et Financière~;
    \item Direction de la Stratégie et du Développement~;
    \item Services Régionaux~:
    \begin{itemize}
        \item SPIE Ouest-Centre~;
        \item SPIE Sud-Ouest~;
        \item SPIE Ile-de-France Nord-Ouest~;
        \item SPIE Est~;
        \item SPIE Sud-Est~;
        \item SPIE Europe du nord~:
        \begin{itemize}
            \item SPIE UK~;
            \item SPIE Nederland~;
            \item SPIE Belgium.
        \end{itemize}
    \end{itemize}
    \item Services de Spécialités~:
    \begin{itemize}
        \item SPIE Communications~;
        \item SPIE Oil \& Gas Service~;
        \item SPIE Nucléaire.
    \end{itemize}
\end{itemize}

\subsection{SPIE Sud-Est}

Dans le cadre du projet, nous nous intéresserons seulement au service régional SPIE SUD EST. Ce dernier
est composé de 3 directions de spécialités qui dépendent eux aussi de la Direction Générale~:

\begin{itemize}
    \item Génie climatique~;
    \item Industries~;
    \item Systèmes d’information \& transport.
\end{itemize}


La direction des Systèmes d’information \& transport assure le déploiement et l’exploitation des systèmes de
gestion~: Gestion des affaires et des moyens; Gestion des RH et payes; Comptabilité; Trésorerie ... De nombreux
progiciels sont intégrés au SI de SPIE.

\subsection{Exemples de contrat de maintenance}

SPIE réalisant 66\% de son chiffre d'affaire en maintenance, nous décrivons ici le fonctionnement des contrats de
maintenance. Ceux-ci sont donc composés de deux parties distinctes~:

\begin{description}
    \item[Partie forfaitaire~:] Concerne la maintenance préventive ou curative régulière des équipements
    après installation. Exemple, la maintenance de l'éclairage publique d'une ville.

    \item[Partie bon de commande~:] Concerne la maintenance curative exceptionnelle suite à un accident, vendalisme et
    même les travaux induits comme l'amélioration des installations ou le traitement de l’obsolescence.
\end{description}


\pagebreak
\section{Positionnement dans le cycle général du développement des SI}

SPIE est malheureusement équipé d'une variété de progiciels pour chacune des tâches, d'où son intention
de changer le tout pour un ERP unique (SAP).

\begin{figure}[h]
    \centering
    \includegraphics[width=140mm]{./images/A_SI_actuelles.png}
    \caption{Cartographie générale du SI de SPIE}
    \label{diagram:si_map}
\end{figure}

Les attentes du clients sont~:

\begin{itemize}
    \item Pour la solution standard : évoluer vers un ERP unique (SAP)
    \item Pour les opérations de maintenance : souhait de saisir les événements et les compte-rendus à la
    source (nomadisme)
\end{itemize}


	\chapter{Description des livrables}

Plusieurs livrables doivent être livrés aux clients tout au long des différentes étapes de l'étude préalable :

	\section{Initialisation et Organisation du projet}

		Deux livrables sont à rendre lors de cette étape afin de fixer le contexte et l'objet du projet et assurer des méthodes et des processus qualité robustes et ainsi assurer des livrables de qualité pour les prochaines étapes.

		\subsection{Dossier d'initialisation}

		Le dossier d'initialisation sert de point de départ pour le projet et doit contenir les éléments suivants :

		\begin{enumerate}
			\item Objet et contexte du projet
			\item Description des livrables attendus
			\item Modes opératoires et phasage
			\item Planning et définition des tâches
			\item Organsation de l'équipe
		\end{enumerate}

		\subsection{Plan d'Assurance Qualité (PAQ)}

		Ce document permet la mise en place d'une politique qualité pour le projet. Il doit contenir :

		\begin{enumerate}
			\item Spécification de la forme (plan, mise en forme) des documents
			\item Cycle de vie des documents
			\item Ressources et outils utilisés
			\item Politique qualité et méthodologie de validation interne et de recette
		\end{enumerate}


	\section{Expression des besoins}

		Cette étape permet de produire, en collaboration avec les clients, le dossier d'expression des besoins.

		\subsection{Dossier d'expression des besoins}

		 Ce dossier résult du receuil les besoins clients et l'étude de l'existant / benchmarking. Il contient :

		\begin{itemize}
		    \item une présentation du contexte du projet (approche métier),
		    \item les éventuelles orientations stratégiques de la MOA
		    \item une analyse de l’existant (dont le SI)
		    \item la cible fonctionnelle (modèle de référence des activités et processus de l’entreprise).
		    \item les écarts avec l’existant (les dysfonctionnements)
		    \item les attentes des partenaires
		    \item le benchmarking
		    \item les thèmes de progrès
		\end{itemize}

	\section{Expression des besoins}

		Cette étape permet de produire, en collaboration avec les clients, le dossier d'expression des besoins.

		Ce dossier résult du receuil les besoins clients et l'étude de l'existant / benchmarking. Il contient :
		\begin{itemize}
		    \item Présentation du contexte Métier du projet
		    \item Orientations stratégiques de la MOA
		    \item Analayse de l’existant
		    \item Cible fonctionnelle : activités et processus de l’entreprise
		    \item Écarts entre la cible et l’existant (manques, dysfonctionnements)
		    \item Attentes des partenaires
		    \item Benchmarking : étude des solutions utilisées dans 2 entreprises du secteur
		    \item Thèmes de progrès
		\end{itemize}


	\section{Élaboration des scénarios}

		Durant cette étape doivent être étudiés et envisagés deux scénarios de mise en œuvre : une solution spécifique et une solution ERP (standard).

		Ce document doit contenir (pour chaque scénario):
		\begin{itemize}
		    \item Nouvelle organisation
		    \item Architecture technique
		    \item Architecture applicative
		    \item Architecture logicielle
		\end{itemize}


	\section{Évaluation des scénarios}

		Les deux scénarios établis précédemment sont évalués lors de cette étape et leurs principaux avantages/inconvénients sont comparés afin de faire un choix.

		Un livrable devra reprendre cette étude en explicitant le choix effectué et en permettant au client de comprendre les raisons en quoi ce dernier répond à ses besoins.nts de choix, à savoir les points forts et les points faibles.


	\section{Restitution}

		Lors de la restitution, plusieurs livrables sont à prévoire: un dossier bilan et une présentation orale (type Powerpoint).

		\subsection{Présentation orale}

		Nous exposerons durant cette présentation les deux solutions étudiées (solution spécifique et solution ERP) et expliquerons notre choix, le but étant de convaincre le client du sérieux de notre étude préalable et de la qualité des livrables fournis.

		\subsection{Dossier bilan}

		Le dossier bilan est une synthèse de l'étude préalable et doit contenir:

		\begin{itemize}
		    \item Les changements qu'ont subi les livrables de la phase d'initialisation à la restitution
		    \item Synthèse des difficultés rencontrées.
		    \item Planning mis à jour
		    \item Justification des écarts entre le planning prévisionnel et le planning effectif.
		\end{itemize}



	\section{Documents de suivi}

		Un certain nombre de documents sont réalisés tout au long de la réalisation de l'étude préalable et sont utiles en interne pour superviser la réalisation des tâches et l'état de l'équipe.

		\begin{itemize}
		    \item Fiche de suivi individuel par séance
		    \item Fiche de suivi global par séance
		    \item Fiche de suivi d’avancement des livrables intermédiaires
		    \item Journal de réunion
		    \item Tableau de bord
		\end{itemize}
	% REMOVE THIS
\documentclass[a4paper,11pt]{report} 

\title{Dossier d'Initialisation}

\date\today

\begin{document}
\maketitle
\newpage

% END REMOVE THIS

\chapter*{M\'ethodes et phasage}
\addcontentsline{toc}{chapter}{M\'ethodes - Modes op\'eratoires - Phasage}
\chaptermark{M\'ethodes - Modes op\'eratoires - Phasage}

\subsection*{M\'ethodes utilis\'ees}

M\'ethodes...

\subsection*{Phases}

Nous diviserons ce projet en 4 grandes phases :


\begin{itemize}
 \item Organisation du projet
 \item Expression des besoins
 \item Construction des solutions
 \item \'Evaluation des sc\'enarii
\end{itemize}


Chacune de ces phases comporte des objectifs pr\'ecis.

\subsubsection*{Organisation du projet}

Afin d'effectuer ce projet dans les meilleures conditions, il est primordial d'adopter une organisation efficace.
Dans un premier temps, il est important de resituer l'\'etude dans son contexte. 
Ceci est n\'ecessaire afin de ne pas perdre de temps \`a \'elaborer des solutions qui s'avèreront \^etre hors du champs de l'\'etude demand\'ee.
Notre \'equipe prendra donc connaissance du projet finement et identifiera sa place dans les activit\'es de l'entreprise.

L'ensemble des livrables \`a fournir devra \^etre identifi\'e afin de d\'eterminer les diff\'erentes t\^aches \`a r\'epartir au sein de l'\'equipe.
Nous identifierons ensuite les contraintes et les risques li\'es \`a ce projet et \'etablirons des plans d'actions pour les g\'erer.

Afin de s'assurer de la qualit\'e des livrables, un plan d'assurance qualit\'e (PAQ) sera mis en place. Dans celui-ci figurera la gestion de la documentation du projet,
le workflow, les proc\'edures de validations internes et externes, ainsi que l'ensemble des outils que nous utiliserons.

\subsubsection*{Expression des besoins}

Cette phase d'expression des besoins consiste \`a faire tout d'abord une \'etude de l'existant.
\subsubsection*{Construction des solutions}
\subsubsection*{\'Evaluation des sc\'enarii}

A la suite de ces phases, nous soutiendrons ce projet devant un jury.

\subsection*{R\'esultats des phases}

Chaque phase engendre des livrables.

% REMOVE THIS
\end{document}
% END REMOVE THIS

	\chapter{Identification des activités et des tâches}

Nous allons, dans cette partie, identifier les activités et les tâches nécassaires à la réalisation des livrables attendus pour la phase d'étude préalable. A chaque tâche, correspondant pour la plupart à 2h de travail par semaine, sera ensuite affectée une ressource (des fois plusieurs). Les tâches seront enfin ordonnancées en fonction des échéances et des disponibilités des ressources. Nous considérons que la charge de travail par ressource est, en moyenne, de 5 heures par semaine.

\section{Liste des tâches et estimation des charges}

\begin{enumerate}
  \item Sous-phase d'initialisation
          \begin{enumerate}
            \item Organisation de l'étude :
              \begin{enumerate}
                  \item formaliser le cadre et le périmètre du projet (2h)
                  \item  recenser les objectifs, les contraintes et les risques de l'étude (2h)
                  \item  organiser l'équipe (distribution des rôles et des responsabilités) (2h)
                  \item  planifier la communication (30m)
                  \item  définir les livrables (1h)
              \end{enumerate}
            \item Planification de l'étude : 
              \begin{enumerate}
                  \item  découper en phases et sous-phases l'étude (1h)
                  \item  identifier les livrables intermédiaires (1h)
                  \item  définir les tâches (2h)
                  \item  évaluer les charges des tâches (2h)
                  \item  répartir et ordonnancer les tâches (2h)
              \end{enumerate}

            \item Planification des formations (30m):
                \begin{enumerate}
                  \item identifier et planifier les formations et leur affecter des ressources
                \end{enumerate}
            \item Choix techniques : 
                \begin{enumerate}
                  \item choisir les méthodes et les outils (2h)
                \end{enumerate}

            \item Prévention :
                \begin{enumerate}
                  \item diffuser le PAQ (2h)
                  \item définir le référenciel d'évaluation interne et externe (2h)
                  \item mettre en place un plan qualité (2h)
                \end{enumerate}

            \item Rédaction des livrables : 
                \begin{enumerate}
                  \item rédiger le dossier d'initialisation 
                  \item rédiger le PAQ 
                \end{enumerate}

            \item Contrôle :
              \begin{enumerate}
                \item revue des livrables (2h)
                \item réunion d'organisation et point sur le travail (1h) 
                \item rédaction des fiches de suivi et du tableau de bord (1h)
              \end{enumerate}
      \end{enumerate}

    \item Sous-phase d'expression des besoins
          \begin{enumerate}
            \item Formations 
                \begin{enumerate}
                  \item ARIS (1h)
                  \item SAP (1h)
                \end{enumerate}
            \item Définition du contexte et du périmètre de l'étude :
                \begin{enumerate}
                  \item décrire les sous-processus à analyser : (3h)
                    \begin{itemize}
                      \item offre et revue d'offre 
                      \item commandde et revue de commande
                      \item lancement des prestations de service et travaux
                      \item réalisation de prestations de maintenance
                      \item réalisation travaux induits
                      \item évolution du contrat
                      \item solde de l'affaire et du contrat 
                    \end{itemize}
                  \item modèles organisationnel, communicationnel, procédural (2h)
                  \item tableau croisé Services/Activités (2h) 
                \end{enumerate}
            \item Diagnostique du SI actuel du point de vue organisationnel : 
                \begin{enumerate}
                  \item décrire l'organisation actuelle du SI (rôles, responsabilités et activités réalisées) (2h)
                  \item représenter les processus et les procédures (2h)
                  \item identifier les dysfonctionnements et les écarts (2h)
                \end{enumerate}
            \item Diagnostique du SI actuel du point de vue informatique : 
                \begin{enumerate}
                  \item décrire l'architecture applicative (2h)
                  \item décrire l'architecture technique (2h)
                  \item cartographie applicative (2h)
                  \item cartographie technique (2h)
                  \item rédiger les fiches application (2h)
                  \item rédiger une synthèse des points forts et des points faibles (2h)
                \end{enumerate}
            \item Analyse des solutions des entreprises leader dans le domaine : 
                \begin{enumerate}
                  \item identifier les entreprises à analyser et pour quel domaine (2h)
                  \item étudier les progiciels de gestion intégrés (2h)
                  \item comprendre les méthodes et prendre connaîssance des indicateurs des entreprises identifiées (quantités, délais, coûts) (2h)
                  \item se situer par rapport aux progiciels (2h)
                  \item se situer par rapport aux performances des entreprises (2h)
                  \item capitaliser leurs meilleures pratiques (2h)
                \end{enumerate}
            \item Elaboration de la cible fonctionnelle : 
                \begin{enumerate}
                  \item créer les modèles de processus et d'activités (2h)
                  \item créer les modèles d'organisation type (2h)
                  \item créer les modèles généraux d'objets métiers à gérer (2h)
                  \item créer les diagrammes types de cas d'utilisation (2h)
                \end{enumerate}
            \item Identification des thèmes de progrès : 
                \begin{enumerate}
                  \item décliner la stratégie de l'entreprise au niveau des processus existants (2h)
                  \item adapter la logique des processus en s'appuyant sur l'architecture de référence (2h)
                  \item adapter l'organisation des acteurs impliqués en fonction des principes d'organisation de la référence (2h)
                  \item identifier les nouvelles technologies à forte valeur ajoutée (2h)
                \end{enumerate}
            \item Composition des livrables : 
                \begin{enumerate}
                  \item rapport de synthèse de l'étude de l'existant 
                  \item rapport de Benchmarking
                  \item rapport de modélisation de la cible fonctionnelle
                \end{enumerate}
            \item Contrôle :
              \begin{enumerate}
                \item revue des livrables (6h)
                \item réunion d'organisation et point sur le travail (3h) 
                \item rédaction des fiches de suivi et du tableau de bord (3h)
              \end{enumerate}
      \end{enumerate}

    \item Sous-phase de construction des solutions

          \begin{enumerate}
            \item Description des impacts sur l'organisation : 
                \begin{enumerate}
                  \item identifier les changements organisationnels (2h)
                  \item mesurer les risques induits des changements des solutions :
                    \begin{itemize}
                      \item standard (1h)
                      \item spécifique (1h)
                    \end{itemize}
                  \item dimensionner les actions à conduire dans les étapes ultérieures pour la solution :
                    \begin{itemize}
                      \item standard (1h)
                      \item spécifique (1h)
                    \end{itemize}
                \end{enumerate}
            \item Analyse de l'architecture applicative cible : 
                \begin{enumerate}
                  \item identifier les paquetages et les classes importantes de l'analyse et les exigences particulières à satisfaire (2h)
                  \item identifier les échanges entre paquetages et les interfaces disponibles (2h)
                \end{enumerate}
            \item Définition des stratégies d'informatisation : 
                \begin{enumerate}
                  \item rechercher les hypothèses envisageables (6h)
                  \item prioriser les hypothèses identifiées (2h)
                  \item concevoir et dimensionner les parties du système, l'architecture logique d'ensemble et l'architecture technique pour la première hypothèse (2h)
                  \item concevoir et dimensionner les parties du système, l'architecture logique d'ensemble et l'architecture technique pour la deuxième hypothèse (2h)
                \end{enumerate}
            \item Conception architecturale logique et technique : 
                \begin{enumerate}
                  \item identifier les solutions progiciel candidates (6h)
                  \item analyser les écarts fonctionnels avec les besoins utilisateurs (2h)  
                  \item identifier les solutions pour traiter les écarts fonctionnels (2h)
                  \item faire un mapping sur l'architecture applicative cible (2h)
                \end{enumerate}
            \item Composition des livrables : 
                \begin{enumerate}
                  \item rédiger le rapport de spécification d'une solution spécifique (8h)
                  \item rédiger le rapport de configuration des scénarii SAP sélectionnés (4h)
                  \item créer les matrices ARIS SAP / fonction SPIE SE et SAP / organigramme SPIE SE (3h)
                  \item générer grâce à ARIS le rapport de modélisation de la solution sélecionnée (30min)
                \end{enumerate}
            \item Contrôle :
              \begin{enumerate}
                \item revue des livrables (4h)
                \item réunion d'organisation et point sur le travail (2h) 
                \item rédaction des fiches de suivi et du tableau de bord (2h)
              \end{enumerate}
      \end{enumerate}

    \item Sous-phase d'élaboration, évaluation et choix des scénarii

          \begin{enumerate}
            \item Identification des scénarii de mise en oeuvre :
                \begin{enumerate}
                  \item créer un scénario pour la solution standard (2h)
                  \item créer un scénario pour la solution spécifique (2h)
                  \item planifier la mise en oeuvre de la solution standard (2h)
                  \item planifier la mise en oeuvre de la solution spécifique (2h)
                \end{enumerate}
            \item Composition des livrables :
                \begin{enumerate}
                  \item rédiger le dossier de choix pour le Comité de Pilotage (6h)
                \end{enumerate}
            \item Contrôle :
              \begin{enumerate}
                \item revue des livrables (2h)
                \item réunion d'organisation et point sur le travail (1h) 
                \item rédaction des fiches de suivi et du tableau de bord (1h)
              \end{enumerate}
      \end{enumerate}

    \item Sous-phase de bilan

          \begin{enumerate}
            \item Préparation des bilans qualitatif et quantitatif :
                \begin{enumerate}
                  \item comparer le contenu du dossier d'initialisation et les livrables effectivement rendus (2h)
                  \item estimer les charges effectives et expliquer les eventuelles différences avec le plan de charge (2h)
                  \item faire un bilan humaine du travail de l'équipe et de ses membres (2h)
                \end{enumerate}
            \item Composition des livrables :
                \begin{enumerate}
                  \item rédiger le dossier de bilan (4h)
                  \item préparer une présentation powerpoint du projet (8h)
                \end{enumerate}
            \item Contrôle :
              \begin{enumerate}
                \item revue des livrables (2h)
                \item réunion d'organisation et point sur le travail (1h) 
                \item rédaction des fiches de suivi et du tableau de bord (1h)
              \end{enumerate}
          \end{enumerate}

\end{enumerate}

\section{Diagramme de GANTT (version 0)}

\begin{figure}[h]
    \centering
    \includegraphics[scale=0.8]{images/Gantt_1.png}
    \caption{Sous-phase d'initialisation}
    \label{diagram:si_map}
\end{figure}

\begin{figure}[h]
    \centering
    \includegraphics[width=150mm]{images/Gantt_2.png}
    \caption{Sous-phase d'expression des besoins}
    \label{diagram:si_map}
\end{figure}

\begin{figure}[h]
    \centering
    \includegraphics[width=150mm]{images/Gantt_3.png}
    \caption{Sous-phase de construction des solutions}
    \label{diagram:si_map}
\end{figure}

\begin{figure}[h]
    \centering
    \includegraphics[scale=0.65]{images/Gantt_4.png}
    \caption{Sous-phase d'élaboration, évaluation et choix des scénarii}
    \label{diagram:si_map}
\end{figure}

\begin{figure}[h]
    \centering
    \includegraphics[scale=0.6]{images/Gantt_5.png}
    \caption{Sous-phase de bilan}
    \label{diagram:si_map}
\end{figure}

	\chapter{Organisation de l'équipe}
Cette partie a pour but de présenter les différents rôles de l'équipe projet que nous avons constituée. Pour chacun des rôles, nous présenterons leurs attributions.

\section{Chef de projet~: Valentina~\textsc{Zantedeschi}}
Chargée de la planification et de l'organisation, elle est également garante de la coordination et de la cohésion du groupe.

Elle effectuera le \textbf{suivi stratégique} du projet en évaluant les risques et en s'assurant du respect des objectifs dans les délais impartis. Elle veillera au bon \textbf{pilotage opérationnel} en planifiant les tâches et en les encadrant. Elle mettra tout en œuvre pour garantir une \textbf{organisation humaine} efficace en définissant le rôle de chacun des membres, en leur attribuant des tâches et en résolvant les éventuels conflits. Enfin, elle s'assurera du bon \textbf{pilotage de la production}, en suivant les résultats et les livrables.

\section{Responsable qualité~: Marina~\textsc{Julien}}
D'une manière générale, elle veillera à la \textbf{qualité des livrables} fournis au client. Elle s'assurera également de la cohésion entre les livrables, de la maintenance et du \textbf{respect du Plan d'Assurance Qualité}.

\section{Expert méthode et outils~: Thierry \textsc{Cantenot}}
Il sera chargé de \textbf{conseiller}, d'\textbf{assister}, de \textbf{former} et d'\textbf{informer}. Il effectuera également la \textbf{veille technologique} quant aux outils utilisés et \textbf{proposera des évolutions} qu'il juge nécessaires.

\section{Expert métier~: Guillaume \textsc{Abadie}}
Il enrichira l'équipe de sa connaissance métier de l'entreprise \texttt{SPI} et, par sa fine compréhension, permettra l'adéquation de la solution conçue aux besoins de l'entreprise. Il est le référent par excellence au sein de notre équipe pour toutes les questions à propos de l'entreprise \texttt{SPI}.


\section{Expert ERP~: Ahmed~:\textsc{Kachkach}}

\section{Responsable communication~: Martin~\textsc{Wetterwald}}
Canal de communication principal, il sera l'interface privilégiée entre l'équipe et les clients, en plus de garantir une bonne communication intra-équipe. Il sera le premier à être contacté par les clients en cas de problème avec les livrables (difficulté de transmission ou non-validation de la part du client).



	\include{F_analyse_risques}

	%\renewcommand{\chaptermark}[1]{\markboth{\MakeUppercase{#1}}{}}
	%\renewcommand{\sectionmark}[1]{\markright{#1}}

	%\addcontentsline{toc}{part}{Annexes}
	%\part*{Annexes}
	%\appendix
	%\include{implementationExercices}


    % ------------------------------------------------------------------- FOOTER
\end{document}
