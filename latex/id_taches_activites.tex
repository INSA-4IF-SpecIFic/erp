\chapter*{Partie D - Identification des activités et des tâches}
\addcontentsline{toc}{chapter}{Identification des activités et des tâches}
\chaptermark{Partie D}

\section*{Liste des activités et des tâches}
\addcontentsline{toc}{section}{Liste des activités et des tâches}

% intro : définition de activité, tâche et ressource

\begin{enumerate}
  \item Phase d'initialisation
          \begin{enumerate}
            \item Organisation de l'étude :
              \begin{enumerate}
                  \item formaliser le cadre et le périmètre du projet 
                  \item  recenser les objectifs, les contraintes et les risques de l'étude 
                  \item  organiser l'équipe (distribution des rôles et des responsabilités) 
                  \item  planifier la communication 
                  \item  définir les livrables 
              \end{enumerate}
            \item Planification de l'étude : 
              \begin{enumerate}
                  \item  découper en phases et sous-phases l'étude 
                  \item  identifier les livrables intermédiaires 
                  \item  définir les tâches 
                  \item  évaluer les charges des tâches 
                  \item  répartir et ordonnancer les tâches 
              \end{enumerate}
          \end{enumerate}

          \begin{enumerate}
            \item Formations éventuelles :
                \begin{enumerate}
                  \item identifier et planifier les formations et leur affecter des ressources
                \end{enumerate}
            \item Expertise technique : 
                \begin{enumerate}
                  \item choisir les méthodes et les outils 
                \end{enumerate}
          \end{enumerate}

          \begin{enumerate}
            \item Prévention :
                \begin{enumerate}
                  \item diffuser le PAQ 
                  \item définir le référenciel d'évaluation interne et externe 
                  \item mettre en place un plan qualité 
                \end{enumerate}
            \item Contrôle :
                \begin{enumerate}
                  \item évaluer la qualité de la sous-phase
                \end{enumerate}
          \end{enumerate}

          \begin{enumerate}
            \item Rédaction des livrables : 
                \begin{enumerate}
                  \item rédiger le dossier d'initialisation 
                  \item rédiger le PAQ 
                \end{enumerate}
          \end{enumerate}

    \item Phase d'expression des besoins
          \begin{enumerate}
            \item Formations ARIS
          \end{enumerate}

          \begin{enumerate}
            \item Contrôle :
                \begin{enumerate}
                  \item revue intermédiaire 
                  \item revue finale 
                  \item évaluer la qualité de la sous-phase 
                \end{enumerate}
          \end{enumerate}

          \begin{enumerate}
            \item Définition du contexte et du périmètre de l'étude :
                \begin{enumerate}
                  \item créer le modèle des activités concernées dans l'entreprise
                  \item  identifier des processus et des procédures à analyser
                \end{enumerate}
            \item Diagnostique du SI actuel du point de vue organisationnel : 
                \begin{enumerate}
                  \item décrire de l'organisation actuelle du SI (rôles, responsabilité et activités réalisées) -> tableau croisé Services/Activités 
                  \item représenter des processus et des procédures -> modèles organisationnel, communicationnel, procédural 
                  \item identifier des dysfonctionnements et des écarts
                \end{enumerate}
            \item Diagnostique du SI actuel du point de vue informatique : 
                \begin{enumerate}
                  \item décrire l'architecture applicative -> cartographie applicative
                  \item décrire l'architecture technique -> cartographie technique 
                  \item rédiger les fiches application
                  \item rédiger une synthèse des points forts et des points faibles
                \end{enumerate}
            \item Analyse des solutions des entreprises leader dans le domaine : 
                \begin{enumerate}
                  \item identifier des entreprises à analyser et pour quel domaine 
                  \item comprendre des leurs méthodes et prise de connaîssance des leurs indicateurs (quantités, délais, coûts) 
                  \item se situer par rapport à leurs performances
                  \item capitaliser leurs meilleures pratiques
                \end{enumerate}
            \item Elaboration de la cible fonctionnelle : 
                \begin{enumerate}
                  \item créer les modèles de processus et d'activités
                  \item créer les modèles d'organisation type
                  \item créer les modèles généraux d'objets métiers à gérer
                  \item créer les diagrammes types de cas d'utilisation
                \end{enumerate}
            \item Identification des thèmes de progrès : 
                \begin{enumerate}
                  \item décliner la stratégie de l'entreprise au niveau des processus existants 
                  \item adapter la logique des processus en s'appuyant sur l'architecture de référence
                  \item adapter l'organisation des acteurs impliqués en fonction des principes d'organisation de la référence
                  \item identifier les nouvelles technologies à forte valeur ajoutée
                \end{enumerate}
            \item Composition des livrables : 
                \begin{enumerate}
                  \item rapport de synthèse de l'étude de l'existant
                  \item ensemble de modèles ARIS 
                  \item rapport de Benchmarking
                  \item rapport de modélisation de la cible fonctionnelle
                \end{enumerate}
          \end{enumerate}

\item Phase de construction des solutions
          \begin{enumerate}
            \item Contrôle :
                \begin{enumerate}
                  \item revue intermédiaire
                  \item revue finale
                  \item évaluer la qualité de la sous-phase
                \end{enumerate}
          \end{enumerate}

          \begin{enumerate}
            \item Description des impacts sur l'organisation : 
                \begin{enumerate}
                  \item identifier les changements organisationnels et en mesurer les risques induits
                  \item dimensionner les actions à conduire dans les étapes ultérieures
                \end{enumerate}
            \item Analyse de l'architecture applicative cible : 
                \begin{enumerate}
                  \item identifier les paquetages et les classes importantes de l'analyse et les exigences particulières à satisfaire 
                  \item identifier les échanges entre paquetages et les interfaces disponibles 
                \end{enumerate}
            \item Définition des stratégies d'informatisation : 
                \begin{enumerate}
                  \item rechercher les hypothèses envisageables 
                  \item prioriser le hypothèses identifiées
                  \item pour chaque hypothèse envisagée, concevoir et dimensionner les parties du système au moyen d'un progiciel et d'un développement spécifique, l'architecture logique d'ensemble et l'architecture technique
                \end{enumerate}
            \item Conception architecturale logique et technique : 
                \begin{enumerate}
                  \item identifier les solutions progiciel candidates 
                  \item pour chaque solution, analyser les écarts fonctionnels avec les besoins utilisateurs, identifier les solutions pour traiter ces écarts, en faire un mapping sur l'architecture applicative cible
                \end{enumerate}
          \end{enumerate}

    \item Phase de construction des solutions
          \begin{enumerate}
            \item Contrôle :
                \begin{enumerate}
                  \item revue intermédiaire
                  \item revue finale
                  \item évaluer la qualité de la sous-phase
                \end{enumerate}
          \end{enumerate}

          \begin{enumerate}
            \item Description des impacts sur l'organisation : 
                \begin{enumerate}
                  \item identifier les changements organisationnels et en mesurer les risques induits
                  \item dimensionner les actions à conduire dans les étapes ultérieures
                \end{enumerate}
            \item Analyse de l'architecture applicative cible : 
                \begin{enumerate}
                  \item identifier les paquetages et les classes importantes de l'analyse et les exigences particulières à satisfaire 
                  \item identifier les échanges entre paquetages et les interfaces disponibles 
                \end{enumerate}
            \item Définition des stratégies d'informatisation : 
                \begin{enumerate}
                  \item rechercher les hypothèses envisageables 
                  \item prioriser le hypothèses identifiées
                  \item pour chaque hypothèse envisagée, concevoir et dimensionner les parties du système au moyen d'un progiciel et d'un développement spécifique, l'architecture logique d'ensemble et l'architecture technique
                \end{enumerate}
            \item Conception architecturale logique et technique : 
                \begin{enumerate}
                  \item identifier les solutions progiciel candidates 
                  \item pour chaque solution, analyser les écarts fonctionnels avec les besoins utilisateurs, identifier les solutions pour traiter ces écarts, en faire un mapping sur l'architecture applicative cible
                \end{enumerate}
            \item Composition des livrables : 
                \begin{enumerate}
                  \item rédiger le rapport de spécification d'un solution spécifique 
                  \item rédiger le rapport de configuration des scénarii SAP sélectionnés
                  \item créer les matrices ARIS SAP / fonction SPIE SE et SAP / organigramme SPIE SE
                  \item générer grâce à ARIS le rapport de modélisation de la solution sélecionnée
                \end{enumerate}
          \end{enumerate}

    \item Phase d'élaboration, évaluation et choix des scénarii

          \begin{enumerate}
            \item Contrôle :
                \begin{enumerate}
                  \item revue finale
                  \item évaluer la qualité de la sous-phase
                \end{enumerate}
          \end{enumerate}

          \begin{enumerate}
            \item Identification des scénarii de mise en oeuvre : 
                \begin{enumerate}
                  \item créer un scénario pour chaque solution envisagée
                  \item planifier la mise en oeuvre de la solution
                \end{enumerate}
            \item Composition des livrables : 
                \begin{enumerate}
                  \item rédiger le dossier de choix pour le Comité de Pilotage
                \end{enumerate}
          \end{enumerate}

\item Phase de bilan

          \begin{enumerate}
            \item Contrôle :
                \begin{enumerate}
                  \item revue finale
                  \item évaluer la qualité de la sous-phase
                \end{enumerate}
          \end{enumerate}

          \begin{enumerate}
            \item Préparation des bilans qualitatifs et quantitatif : 
                \begin{enumerate}
                  \item comparer le contenu du dossier d'initialisation et les livrables effectivement rendus
                  \item estimer les charges effectives et expliquer les eventuelles différences avec le plan de charge
                  \item faire un bilan humaine du travail de l'équipe et de ses membres
                \end{enumerate}
            \item Composition des livrables : 
                \begin{enumerate}
                  \item rédiger le dossier de bilan
                  \item préparer une présentation powerpoint du projet
                \end{enumerate}
          \end{enumerate}

\end{enumerate}