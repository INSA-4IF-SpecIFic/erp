\chapter*{Partie D - Identification des activités et des tâches}
\addcontentsline{toc}{chapter}{Identification des activités et des tâches}
\chaptermark{Partie D}

\section*{Liste des activités et des tâches}
\addcontentsline{toc}{section}{Liste des activités et des tâches}

% intro : définition de activité, tâche et ressource

\begin{enumerate}
  \item Phase d'initialisation
    \begin{description}

      \item[Activités de conduite du projet] \hfill \\
          \begin{enumerate}
            \item Organisation de l'étude :
              \begin{enumerate}
                  \item formaliser le cadre et du périmètre du projet 
                  \item  recenser les objectifs, des contraintes et des risques de l'étude 
                  \item  organiser l'équipe (distribution des rôles et des responsabilités) 
                  \item  planifier la communication 
                  \item  définir les livrables 
              \end{enumerate}
            \item Planification de l'étude : 
              \begin{enumerate}
                  \item  découper en phases et sous-phases l'étude 
                  \item  identifier les livrables intermédiaires 
                  \item  définir les tâches 
                  \item  évaluer les charges des tâches 
                  \item  répartir et ordonnancer les tâches 
              \end{enumerate}
          \end{enumerate}

      \item[Activités de support du projet] \hfill \\
          \begin{enumerate}
            \item Formations éventuelles :
                \begin{enumerate}
                  \item identifier et planifier les formations et leur affecter des ressources
                \end{enumerate}
            \item Expertise technique : 
                \begin{enumerate}
                  \item choisir les méthodes et les outils 
                \end{enumerate}
          \end{enumerate}

      \item[Activités de contrôle du projet] \hfill \\
          \begin{enumerate}
            \item Prévention :
                \begin{enumerate}
                  \item diffuser le PAQ 
                  \item définir le référenciel d'évaluation interne et externe 
                  \item mettre en place un plan qualité 
                \end{enumerate}
            \item Contrôle :
                \begin{enumerate}
                  \item évaluer de la qualité de la sous-phase
                \end{enumerate}
          \end{enumerate}

      \item[Activités de production] \hfill \\
          \begin{enumerate}
            \item Rédaction des livrables : 
                \begin{enumerate}
                  \item rédiger le dossier d'initialisation 
                  \item rédiger le PAQ 
                \end{enumerate}
          \end{enumerate}

    \end{description}

    \item Phase d'expression des besoins
    \begin{description}

      \item[Activités de support du projet] \hfill \\
          \begin{enumerate}
            \item Formations ARIS
          \end{enumerate}

      \item[Activités de contrôle du projet] \hfill \\
          \begin{enumerate}
            \item Contrôle :
                \begin{enumerate}
                  \item revue intermédiaire 
                  \item revue finale 
                  \item évaluer la qualité de la sous-phase 
                \end{enumerate}
          \end{enumerate}

      \item[Activités de production] \hfill \\
          \begin{enumerate}
            \item Définition du contexte et du périmètre de l'étude :
                \begin{enumerate}
                  \item créer le modèle des activités concernées dans l'entreprise
                  \item  identifier des processus et des procédures à analyser
                \end{enumerate}
            \item Diagnostique du SI actuel du point de vue organisationnel : 
                \begin{enumerate}
                  \item décrire de l'organisation actuelle du SI (rôles, responsabilité et activités réalisées) -> tableau croisé Services/Activités 
                  \item représenter des processus et des procédures -> modèles organisationnel, communicationnel, procédural 
                  \item identifier des dysfonctionnements et des écarts
                \end{enumerate}
            \item Diagnostique du SI actuel du point de vue informatique : 
                \begin{enumerate}
                  \item décrire l'architecture applicative -> cartographie applicative
                  \item décrire l'architecture technique -> cartographie technique 
                  \item rédiger les fiches application
                  \item rédiger une synthèse des points forts et des points faibles
                \end{enumerate}
            \item Analyse des solutions des entreprises leader dans le domaine : 
                \begin{enumerate}
                  \item identifier des entreprises à analyser et pour quel domaine 
                  \item comprendre des leurs méthodes et prise de connaîssance des leurs indicateurs (quantités, délais, coûts) 
                  \item se situer par rapport à leurs performances
                  \item capitaliser leurs meilleures pratiques
                \end{enumerate}
            \item Elaboration de la cible fonctionnelle : 
                \begin{enumerate}
                  \item créer les modèles de processus et d'activités
                  \item créer les modèles d'organisation type
                  \item créer les modèles généraux d'objets métiers à gérer
                  \item créer les diagrammes types de cas d'utilisation
                \end{enumerate}
            \item Identification des thèmes de progrès : 
                \begin{enumerate}
                  \item décliner la stratégie de l'entreprise au niveau des processus existants 
                  \item adapter la logique des processus en s'appuyant sur l'architecture de référence
                  \item adapter l'organisation des acteurs impliqués en fonction des principes d'organisation de la référence
                  \item identifier les nouvelles technologies à forte valeur ajoutée
                \end{enumerate}
            \item Composition des livrables : 
                \begin{enumerate}
                  \item rapport de synthèse de l'étude de l'existant
                  \item ensemble de modèles ARIS 
                  \item rapport de Benchmarking
                  \item rapport de modélisation de la cible fonctionnelle
                \end{enumerate}
          \end{enumerate}

    \end{description}

\end{enumerate}