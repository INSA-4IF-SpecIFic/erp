\chapter*{Partie D - Identification des activités et des tâches}
\addcontentsline{toc}{chapter}{Identification des activités et des tâches}
\chaptermark{Partie D}

\section*{Liste des activités et des tâches}
\addcontentsline{toc}{section}{Liste des activités et des tâches}

% intro : définition de activité, tâche et ressource

\begin{enumerate}
  \item Phase d'initialisation
    \begin{enumerate}

      \item Activités de conduite du projet
          \begin{itemize}
            \item Organisation de l'étude :\hfill \\
                  formalisation du cadre du projet \hfill \\
                  organisation de l'équipe (distribution des rôles et des responsabilités) \hfill \\
                  définition des livrables \hfill \\
            \item Planification des tâches : \hfill \\
                  découpage en phases et en sous-phases \hfill \\
                  identification des livrables intermédiaires \hfill \\
                  définition des tâches \hfill \\
                  répartition des tâches et évaluation de leur charge \hfill \\
          \end{itemize}

      \item Activités de support du projet
          \begin{itemize}
            \item Formations éventuelles :\hfill \\
                  identification et planification \hfill \\
            \item Expertise technique : \hfill \\
                  choix des méthodes et des outils \hfill \\
          \end{itemize}

      \item Activités de contrôle du projet
          \begin{itemize}
            \item Prévention :\hfill \\
                  diffusion du PAQ \hfill \\
                  définition du référenciel d'évaluation interne et externe \hfill \\
                  mise en place d'un plan qualité \hfill \\
          \end{itemize}

      \item Activités de production
          \begin{itemize}
            \item Recensement des objectifs, des contraintes et des risques de l'étude :\hfill \\
                  identification et planification \hfill \\
            \item Rédaction des livrables : \hfill \\
                  dossier d'initialisation \hfill \\
                  PAQ \hfill \\
          \end{itemize}
          
    \end{enumerate}

\end{enumerate}